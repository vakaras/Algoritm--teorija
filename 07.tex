\chapter{Primityviai rekursyvios funkcijos}

Kompozicijos operatorius: sakysime, kad funkcija
$f(x_1,x_2,\dotsc,x_n)$ yra gauta iš funkcijų 
$g_{i}(x_1,x_2,\dotsc,x_n)\,(i=1,2,3,\dotsc,m)$ ir funkcijos
$h(x_1,x_2,\dotsc,x_m)$, jei 
$f(x_1,x_2,\dotsc,x_n) = %
h(g_{1}(x_1,\dotsc,x_n),\dotsc,g_{m}(x_1,\dotsc,x_n))$.

\begin{defn}
  Sakysime, jog funkcija $f(x_1,x_2,\dotsc,x_n)$ yra gauta pagal 
  primityviosios rekursijos operatorių iš 
  $g(x_1,x_2,\dotsc,x_{n-1})$ ir 
  $h(x_1,x_2,\dotsc,x_n,x_{n+1})$, jei teisinga:
  \begin{align*}
    f(x_1,x_2,\dotsc,x_{n-1},0) &= g(x_1,x_2,\dotsc,x_{n-1}) \text{ ir} \\
    f(x_1,x_2,\dotsc,x_{n-1},y+1) &= %
      h(x_1,x_2,\dotsc,x_{n-1},y,f(x_1,x_2,\dotsc,x_{n-1},y))
  \end{align*}
\end{defn}

\begin{defn}[Primityviai rekursyvių funkcijų aibė (\emph{PR})]
  Primityviai rekursyvių funkcijų aibė sutampa su aibe, kuriai
  priklauso bazinės funkcijos: 
  $0, s(x) = x+1, pr_{m}^{i}(x_1,x_2,\dotsc,x_m) = x_{i}$ ir
  kuri yra uždara kompozicijos ir primityviosios rekursijos
  operatorių atžvilgiu.
\end{defn}

Visos \emph{PR} funkcijos yra apibrėžtos su visais natūraliaisiais 
skaičiais.

Sakysime, kad $\mathbb{N}$ skaičių poaibis yra primityviai rekursyvus,
jei jo charakteringoji funkcija priklauso \emph{PR}.

\begin{prop}
  Cantaro funkcijos $\alpha_{n}(x_1,x_2,\dotsc,x_n)$ ir 
  $\pi^{i}_{n}(K)$ yra primityviai rekursyvios.
\end{prop}

\begin{prop}
  Jei funkcija $g(x_1,x_2,\dotsc,x_n) \in \PR$, tai ir 
  funkcija
  $f(x_1,x_2,\dotsc,x_n) = \sum _{i=0} ^{x_n} g(x_1,x_2,\dotsc,x_n)%
  \in \PR$
\end{prop}

\begin{prop}
  Jei $f_{i}(x_1,x_2,\dotsc,x_n) \in \PR$ ir 
  $\alpha_{j}(x_1,x_2,\dotsc,x_n) \in \PR$ 
  $(i=1,2,\dotsc,s,s+1;\,j=1,2,\dotsc,s)$, tada ir funkcija:
  \[
  h(x_1,x_2,\dotsc,x_n) =%
  \begin{cases}
    f_{1}(x_1,x_2,\dotsc,x_n), 
      &\text{ jei } \alpha_{1}(x_1,x_2,\dotsc,x_n)=0,\\
    f_{2}(x_1,x_2,\dotsc,x_n), 
      &\text{ jei } \alpha_{2}(x_1,x_2,\dotsc,x_n)=0,\\
    \cdots & {} \\
    f_{s}(x_1,x_2,\dotsc,x_n), 
      &\text{ jei } \alpha_{s}(x_1,x_2,\dotsc,x_n)=0,\\
    f_{s+1}(x_1,x_2,\dotsc,x_n), 
      &\text{ kitu atveju }
  \end{cases}
  \]
  yra primityviai rekursyvi.
  \begin{note}
    Su bet kuriuo rinkiniu $(x_1,x_2,\dotsc,x_n)$ ne daugiau nei viena
    $\alpha_{i}(x_1,x_2,\dotsc,x_n)$ ($i=1,2,\dotsc,s$) gali būti lygi 0.
  \end{note}
\end{prop}

\begin{defn}[Iteracijos operatorius]
  Sakysime, kad funkcija $f(x)$ yra gauta pagal iteracijos operatorių iš
  funkcijos $g(x)$, jei teisinga:
  \[
  \begin{cases}
    f(0) &= 0 \\
    f(y+1) &= g(f(y))
  \end{cases}
  \]
\end{defn}

\begin{defn}
  Visų vieno argumento $\PR$ funkcijų aibė sutampa su aibe, kuriai
  priklauso funkcijos $s(x) = x+1$, 
  $q(x) = x \dotminus \left[ \sqrt{x} \right]^2$ ir kuri yra 
  uždara kompozicijos, sudėties ir iteracijos operatorių atžvilgiu.
\end{defn}

\section{Dalinai rekursyvios funkcijos}

\begin{defn}[Minimizacijos operatorius]
  Sakysime, kad funkcija $f(x_1,x_2,\dotsc,x_n)$ gauta pagal minimizacijos
  operatorių iš funkcijos $g(x_1,x_2,\dotsc,x_n)$, jei:
  \[
  f(x_1,x_2,\dotsc,x_n) = \mu_{y}(g(x_1,x_2,\dotsc,x_{n-1},y) = x_n),
  \]
  tai yra pati mažiausia natūrali $y$ reikšmė, kuriai teisinga
  $g(x_1,x_2,\dotsc,x_{n-1},y) = x_n$.

  \begin{exmp}
    $f(x,y) = \mu_{z}((s(z)\dotminus x)\cdot \sg(y \dotminus z)=x)$
  \end{exmp}
\end{defn}

\begin{defn}[Dalinai rekursyvių funkcijų aibė (DR)]
  Dalinai rekursyvių funkcijų aibė sutampa su aibe, kuriai priklauso 
  bazinės funkcijos $0, s(x) = x+1, pr_{n}^{i}(x_1,x_2,\dotsc,x_n) = x_{i}$
  ir kuri yra uždara kompozicijos, primityviosios rekursijos ir
  minimizacijos operatorių atžvilgiu.

  \begin{note}
    Jei $f(x_1,x_2,\dotsc,x_n) = \mu_{y}(g(x_1,x_2,\dotsc,x_{n-1},y)=x_n)$,
    tai $f(x_1,x_2,\dotsc,x_n)$ reikšmė apskaičiuojama remiantis tokiu
    algoritmu:
    \[
    \verb|for (y = 0; |g(x_1,x_2,\dotsc,x_{n-1},y) %
      \neq x_n\verb|; y++);|      
    \]
    \[
    f(x_1,x_2,\dotsc,x_n) =%
    \begin{cases}
      y, & \text{jei pavyko rasti $y$ arba} \\
      \infty, & \text{jei $y$ neegzistuoja, arba skaičiuojant buvo %
        gauta neapibrėžtis.}
    \end{cases}
    \]
  \end{note}
\end{defn}

\begin{defn}[Bendrųjų reukursyviųjų funkcijų aibė ($\BR$)]
  Aibė sudaryta iš visų $\DR$ funkcijų, kurios yra apibrėžtos su visais
  natūraliaisiais argumentais.
\end{defn}

\begin{prop}
  $\PR \subseteq \BR \subseteq \DR$
\end{prop}

\begin{note}
  Žinomos primityviai rekursyvios funkcijos:
  \begin{itemize}
    \item bazinės funkcijos:
      \begin{align*}
        & 0,\\
        & s(x) = x + 1, \\
        & pr^{i}_{n}(x_1,x_2,\dotsc,x_n) = x_i;
      \end{align*}
    \item vieno argumento $\PR$ funkcijų bazinės funkcijos:
      \begin{align*}
        s(x) &= x + 1, \\
        q(x) &= x \dotminus \left[ \sqrt{x} \right]^{2};
      \end{align*}
    \item kitos žinomos $\PR$ funkcijos:
      \begin{align*}
        \sg(x) &= %
        \begin{cases}
          1, & \text{ jei } x > 0 \\
          0, & \text{ jei } x = 0
        \end{cases}, \\
        \sgi(x) &= %
        \begin{cases}
          0, & \text{ jei } x > 0 \\
          1, & \text{ jei } x = 0
        \end{cases}, \\
        x \dotminus y &=%
        \begin{cases}
          x - y, & \text{ jei } x > y \\
          0, & \text{ kit atveju}
        \end{cases}, \\
        x + y,& \\
        x \cdot y, & \\
        |x - y|, & \\
        [x / y]. & 
      \end{align*}
  \end{itemize}
\end{note}

\begin{note}
  Žinomos dalinai rekursyvios funkcijos:
  \begin{itemize}
    \item dalinis skirtumas:
      \[
      x - y =%
      \begin{cases}
        x - y, & \text{ jei } x \geq y \\
        \infty, & \text{ kitu atveju}
      \end{cases},
      \]
    \item dalinė dalyba:
      \[
      x / y =%
      \begin{cases}
        x / y, & \text{ jei $x$ dalosi iš $y$ }\\
        \infty, & \text{ kitu atveju}
      \end{cases},
      \]
    \item dalinė šaknis:
      \[
      \sqrt{x} =%
      \begin{cases}
        \sqrt{x}, & \text{ jei $x$ yra natūralaus skaičiaus kvadratas} \\
        \infty, & \text{ kitu atveju}
      \end{cases}.
      \]
  \end{itemize}
\end{note}

\section{Reukursyviai skaičiosios aibės}

\begin{defn}[Rekursyviai skaiti aibė]
  \label{rsadr}
  Aibė $A$ yra rekursyviai skaiti, jei sutampa su kažkurios $\DR$ funkcijos
  apibrėžimo sritimi.
\end{defn}

\begin{defn}[Rekursyviai skaiti aibė]
  \label{rsapr}
  Netuščia aibė $A$ yra rekursyviai skaiti, jei sutampa su kažkurios
  $\PR$ funkcijos reikšmių sritimi.
\end{defn}

\begin{defn}[Rekursyviai skaiti aibė]
  \label{rsalg}
  Aibė $A$ yra rekursyviai skaiti, jei $\exists$ tokia $\PR$ funkcija
  $f(a,x)$, kad lygtis $f(a, x) = 0$ turi sprendinį tada ir tik tada, kai
  $a \in A$.
\end{defn}

\begin{prop}
  Visi trys (\ref{rsadr}, \ref{rsapr} ir \ref{rsalg}) rekursyvios aibės
  apibrėžimai yra ekvivalentūs netuščios aibės atžvilgiu.
  \begin{proof}
    (Dalinis \ref{rsapr} $\implies$ \ref{rsadr}, 
    \ref{rsapr} $\implies$ \ref{rsalg} ir 
    \ref{rsalg} $\implies$ \ref{rsapr})

    \begin{description}
      \item[(\ref{rsapr} $\implies$ \ref{rsadr})]
        Tarkime, kad aibė $A$ yra rekursyviai skaiti, nes $\exists \PR$
        funkcija $h(x)$ tokia, kad $A$ sutampa su $h$ reikšmių 
        sritimi (\ref{rsapr} ap.):
        \[
        A = \left\{ h(0), h(1), h(2), h(3), \dotsc \right\}.
        \]
        Imkime funkciją $f(x) = \mu_{y}(h(y) = x)$. Ji yra dalinai 
        rekursyvi, nes yra gauta pagal minimizacijos operatorių iš
        dalinai rekursyvios funkcijos $h (\in \PR \subseteq \DR)$.
        \begin{itemize}
          \item Jei $a \in A$, tai $\exists i(i \in \mathbb{N})$, kad
            $h(i) = a \implies f(a) = \mu_{y}(h(y) = a) < \infty$ 
            – apibrėžta.
          \item Jei $a \not \in A$, tai 
            $\forall j (j \in \mathbb{N}): \, h(j) \neq a \implies%
            f(a) = \mu_{y}(h(y) = a) = \infty$ – neapibrėžta.
        \end{itemize}
        Aibė $A$ sutampa su funkcijos $f(x)$ apibrėžimo sritimi ir
        $f(x) \in \DR \implies$ \ref{rsadr} ap.
      \item[(\ref{rsapr} $\implies$ \ref{rsalg})] 
        Tarkime, jog aibė $A$ sutampa su $\PR$ funkcijos
        $h(x)$ apibrėžimo sritimi (\ref{rsapr} ap.). Tai yra:
        \[
        A = \left\{ h(0),h(1),h(2),h(3),\dotsc \right\}.
        \]
        Imkime funkciją $f(a,x) = |h(x)-a|$. $f(x) \in \PR$, nes gauta
        iš $\PR$ funkcijų pritaikius kompozicijos operatorių. Lygtis
        $f(a,x) = 0$ yra $|h(x) - a| = 0$.
        \begin{itemize}
          \item Jei $a \in A$, tai $\exists$ toks $i$, kad 
            $h(i)=a \implies |h(x) - a| = 0$, turi sprendinį $x = i$.
          \item Jei $a \not \in A$, tai 
            $\forall j (j \in \mathbb{N}): h(j) \neq a \implies %
            |h(x) - a| > 0$ – lygtis neturi sprendinių.
        \end{itemize}
        Lygtis $f(a,x)=0$, turi sprendinį tada ir tik tada, 
        kai $a \in A \implies$ \ref{rsalg} ap.
      \item[(\ref{rsalg} $\implies$ \ref{rsapr})]
        Tarkime, jog $a \in A$ tada ir tik tada, kai lygtis 
        $f(a,x) = 0$, kur $f(a,x) \in \PR$, turi sprendinį.
        Imkime funkciją
        \[
        h(t)=\pi^{1}_{2}(t) \cdot \sgi(f(\pi^{1}_{2}(t),\pi^{2}_{2}(t)))%
        + d \cdot \sg(f(\pi^{1}_{2}(t), \pi_{2}^{2}(t))), 
        \]
        kur $d$ – bet koks aibės $A$ elementas. $h(t) \in \PR$, nes gauta
        iš žinomų $\PR$ funkcijų pritaikius kompozicijos operatorių.
        Parodysime, kad $h(t)$ reikšmių sritis sutampa su aibe $A$:
        \begin{itemize}
          \item Jei $a \in A$, tai lygtis $f(a,x) = 0$ turi sprendinį
            $x = u$. Pažymėkime $t = \alpha_{2}(a, u)$. Tada
            \begin{align*}
              h(t) &= h(\alpha_{2}(a, u)) \\
              &= a \cdot \sgi(\underbrace{f(a, u)}_{= 0}) + %
                d \cdot \sg(\underbrace{f(a, u)}_{= 0}) \\
              &= a \cdot \sgi(0) + d \cdot \sg(0) \\
              &= a
            \end{align*}
          \item Jei $t = \alpha_{2}(a, v)$, kad $f(a, v) \neq 0$, tai
            \begin{align*}
              h(t) &= h(\alpha_{2}(a, v)) \\
              &= a \cdot \sgi(\underbrace{f(a, v)}_{> 0}) + %
                d \cdot \sg(\underbrace{f(a, v)}_{> 0}) \\
              &= a \cdot 0 + d \cdot 1 \\
              &= d
            \end{align*}
        \end{itemize}
        Gavome, kad $h(t)$ su bet kokiu $t$ įgyja reikšmę $\in A$ ir 
        $h(t)$ įgyja visas reikšmes iš aibės $A$. Kadangi 
        $h(t)$ reikšmių sritis yra aibė $A$ ir $h(t) \in \PR$, tai
        gavome \ref{rsapr} apibrėžimą.
    \end{description}
  \end{proof}
\end{prop}

\begin{note}
  Rekursyvių ir rekursyviai skaičių aibių skirtumas:
  \begin{itemize}
    \item $A$ – rekursyvi, jei 
      \[
      \exists \, \chi_{A}(a) =%
      \begin{cases}
        1, & \text{ jei } a \in A \\
        0, & \text{ jei } a \not \in A
      \end{cases}%
      \in \BR
      \]
      (Funkcija visur apibrėžta.)
    \item $A$ – rekursyviai skaiti, jei
      \[
      \exists \, \chi_{A}(a) =%
      \begin{cases}
        1, & \text{ jei } a \in A \\
        \infty, & \text{ jei } a \not \in A
      \end{cases}%
      \in \DR
      \]
      (Funkcija ne visur apibrėžta.)
  \end{itemize}
\end{note}

\subsection{Rekursyviai skaičių aibių savybės}

TODO
