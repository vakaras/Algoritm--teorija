\chapter{$\lambda$-skaičiavimas}

\begin{defn}[$\lambda$–skaičiavimo termas]
  \hfill \\
  \begin{itemize}
    \item Jei $u$ – kintamasis, tai $u$ yra termas.
    \item Jei $E_{1}$ ir $E_{2}$ yra termai, tai ir $(E_{1}E_{2})$ yra 
      termas.
    \item Jei $E$ yra termas, o $x$ – kintamasis, tai ir 
      $\lambda x.E$ irgi yra termas.
  \end{itemize}
  \begin{note}
    Kintamuosius žymime mažosiomis raidėmis.
  \end{note}
  \begin{note}
    Skliaustai būtini: $(xy)z \not = x(yz)$!
  \end{note}
\end{defn}

Kintamojo įeitis terme yra jo pasitaikymas jame.
\begin{exmp}
  Jei turime termą:
  \[
  (\lambda x.((yz)(\lambda z.((zx)y)))(x(\lambda y.(xy)))),
  \]
  tai:
  \[
  (\lambda \underbrace{x}_{1}.((yz)
  (\lambda z.((z\underbrace{x}_{2})y)))
  (\underbrace{x}_{3}(\lambda y.(\underbrace{x}_{4}y)))),
  \]
  sunumeruotos yra kintamojo $x$ įeitys.
\end{exmp}

Kintamojo įeitis $x$ yra suvaržyta, jei patenka į $\lambda x.$ veikimo
sritį. Kitaip $x$ įeitis yra laisva.
\begin{note}
  $\lambda x.E$ terme $\lambda x.$ veikimo sritis yra termas $E$.
\end{note}
\begin{note}
  $\lambda x.xu = (\lambda x.x)u \not = \lambda x.(xu)$ 
\end{note}

Termai $E_{1}$ ir $E_{2}$ yra $\alpha$-ekvivalentūs, jei $E_{2}$ yra
gautas iš $E_{1}$, jame visas laisvas kažkurio kintamojo $x$ įeitis
pakeitus nauju kintamuoju.

Termas $\lambda x.E_{1}$ yra $\alpha$-ekvivalentus termui 
$\lambda y.E_{2}$, jei termas $E_{2}$ yra gautas iš termo $E_{1}$, jame
visas  laisvas $x$ įeitis pakeitus nauju kintamuoju $y$.

\begin{defn}[Redeksas ir jo santrauka]
  Termas pavidalo $(\lambda x.E)Y$, kur $E$ ir $Y$ yra termai, vadinamas
  redeksu. Redekso $(\lambda x.E)Y$ santrauka yra termas $E[^Y/_x]$
  – termas $E$, kuriame visos laisvos kintamojo $x$ įeitys yra pakeistos
  termu $Y$.
  \begin{exmp}
    Redekso $(\lambda x.\underbrace{(ux))}_{E}\underbrace{(zz)}_{Y}$
    santrauka yra termas $u(zz)$.
  \end{exmp}
  \begin{exmp}
    Redekso $(\lambda x.((ux)(\lambda x.(xy))))(zz)$ santrauka yra 
    termas $(u(zz))(\lambda x.(xy))$.
  \end{exmp}
\end{defn}

\begin{defn}[$\beta$-redukcija]
  $\lambda$-skaičiavimo termo $E$ $\beta$-redukcija vadinama termų seka
  $E_1 \triangleright E_2 \triangleright E_3 \triangleright %
  E_4 \triangleright \cdots \triangleright E_k \triangleright$, kur
  $E_1 = E$, ir $E_{i+1}$ yra gautas iš $E_{i}$ jame pirmąjį iš kairės 
  redeksą pakeitus jo santrauka.
\end{defn}

\begin{defn}[Normalinis termas]
  Termas, kuriame nėra redeksų. Termas yra nenormalizuojamas, jeigu jo 
  $\beta$-redukcija yra begalinė.
\end{defn}

\begin{defn}
  $\lambda$-skaičiavimo loginės konstantos yra 
  $\underbrace{\lambda x.\lambda y.y}_{klaidinga}$ ir 
  $\underbrace{\lambda x.\lambda y.x}_{teisinga}$.
  \begin{notation}
    $\lambda x.\lambda y.y = 0$ ir $\lambda x.\lambda y.x = 1$.
  \end{notation}
\end{defn}

\begin{defn}
  $\lambda$-skaičiavimo natūrinis skaičius $k$ yra 
  $\lambda f.\lambda x.(f^{k} x)$, kur
  $f^{k} x = \underbrace{f(f(\dots f(f}_{\text{k kartų}} x )\dots))$.
  \begin{notation}
    Skaičius 2 žymimas 
    $\underline{2} = \lambda f.\lambda x(f^{2} x) =%
    \lambda f.\lambda x(f(fx))$
  \end{notation}
\end{defn}

\begin{exmp}
  $(x1)\underline{1} = (x(\lambda x.\lambda y.x))(\lambda f.\lambda x.(fx))$
\end{exmp}

\begin{defn}
  Sakysime, kad termas $E$ apibrėžia dalinę funkciją 
  $f(x_1,x_2,\dotsc,x_n)$, jei:
  \begin{itemize}
    \item jei $f(K_1,K_2,\dotsc,K_n) = K$, tai termas 
      $(\dots((EK_1)K_2)\dots)K_n$ redukuojamas ($\beta$-redukcijoje) į
      termą $K$;
    \item jei $f(K_1,K_2,\dotsc,K_n) = \infty$, tai termas 
      $(\dots((EK_1)K_2)\dots)K_n$ yra neredukuojamas.
  \end{itemize}
\end{defn}

\begin{prop}
  Kiekvienai algoritmiškai apskaičiuojamai funkcijai egzistuoja ją 
  apibrėžiantis termas.
\end{prop}


