\chapter{Hilberto tipo teiginių skaičiavimas}

\begin{defn}[Hilberto tipo teiginių skaičiavimas]
  Skaičiavimas, kuriame yra aksiomų schemos:
  \renewcommand{\theenumii}{\arabic{enumii}}
  \renewcommand{\labelenumii}{\theenumii}
  \begin{enumerate}
    \item%
      \begin{enumerate}
        \item $A \to (B \to A)$
        \item $(A \to (B \to C)) \to ((A \to B) \to (A \to C))$
      \end{enumerate}
    \item%
      \begin{enumerate}
        \item $(A \land B) \to A$
        \item $(A \land B) \to B$
        \item $(A \to B) \to ((A \to C) \to (A \to (B \land C)))$
      \end{enumerate}
    \item%
      \begin{enumerate}
        \item $A \to (A \lor B)$
        \item $B \to (A \lor B)$
        \item $(A \to C) \to ((B \to C) \to ((A \lor B) \to C))$
      \end{enumerate}
    \item%
      \begin{enumerate}
        \item $(A \to B) \to (\neg B \to \neg A)$
        \item $A \to \neg\neg A$
        \item $\neg\neg A \to A$
      \end{enumerate}
  \end{enumerate}
  ir taisyklė \emph{Modus Ponens (MP)}:
  \[
  \begin{array}{r c l}
    \text{prielaidos} \,\{ & A \qquad A \to B & %
      \text{(MP), čia A, B, C – bet kokios teiginių logikos formulės.} \\
    \cline{2-2}
    \text{išvados} \,\{ & B &
  \end{array}
  \]

\end{defn}

Sakysime, jog formulė $F$ yra įvykdoma Hilberto tipo teiginių skaičiavime,
jeigu galima parašyti tokią formulių seką, kurioje kiekviena formulė yra:
\begin{itemize}
  \item arba aksioma,
  \item arba gauta pagal \emph{MP} iš ankstesnių
\end{itemize}
ir kuri (seka) baigiasi formule $F$.

\begin{prop}
  Jei formulė yra įrodoma Hilberto tipo teiginių skaičiavime, tai ji yra
  tapačiai teisinga ir atvirkščiai (jei nėra įrodoma, tai nėra tapačiai
  teisinga).
\end{prop}

Sakysime, jog skaičiavimo aksioma yra \emph{nepriklausoma}, jei ją išmetus
iš skaičiavimo, ji nėra išvedama jame.

Visos Hilberto teiginių skaičiavimo aksiomos yra nepriklausomos.

\section{Dedukcijos teorema}

Sakysime, kad formulė $F$ yra išvedama iš prielaidų 
$A_{1},A_{2},A_{3},\dots,A_{n}$, jei egzistuoja tokia formulių seka, 
kurioje kiekviena formulė yra:
\begin{itemize}
  \item arba aksioma,
  \item arba prielaida,
  \item arba gauta pagal \emph{MP} iš ankstesnių
\end{itemize}
ir kuri (seka) baigiasi formule $F$.
\begin{notation}
  $A_{1},A_{2},\dots,A_{n} \vdash F$
\end{notation}

\begin{prop}
  (Dedukcijos teorema) $\Gamma \vdash A \to B \iff \Gamma,A \vdash B$, 
  kur $A$,$B$ – bet kokios formulės, $\Gamma$ – baigtinė formulių aibė 
  (gali būti tuščia).
  \begin{proof}
    \hfill \\
    \begin{description}
      \item[Būtinumas] \hfill \\
      Jei $\Gamma \vdash A \to B$, tada egzistuoja formulių seka:
      \[
      \begin{array}{l l l}
        \text{1.} & F_{1}   & {}    \\
        \text{2.} & F_{2}   & {}    \\
        {}        & \cdots  & {}    \\
        \text{i.} & F_{i}   & {}    \\
        {}        & \cdots  & {}    \\
        \text{k.} & F_{k} = A \to B & \\
        \text{k+1.} & A & \text{prielaida} \\
        \text{k+2.} & B & \text{\emph{MP} iš k ir (k+1).} \\
      \end{array}
      \]
      kurioje:
      \[
      F_{i} = \left\{
      \begin{array}{l}
        \text{aksioma} \\
        \text{prielaida iš } \Gamma \\
        \text{gauta pagal \emph{MP} iš ankstesnių}
      \end{array} \right.
      \]
      $\Gamma, A \vdash B$.

      \item[Pakankamumas] \hfill \\
      Jei $\Gamma,A \vdash B$, tai yra tokia formulių seka:
      \[
      \begin{array}{l l}
        \text{1.} & F_{1}     \\
        \text{2.} & F_{2}     \\
        {}        & \cdots    \\
        \text{i.} & F_{i}     \\
        {}        & \cdots    \\
        \text{k.} & F_{k} = B \\
      \end{array}
      \]
      kurioje:
      \[
      F_{i} = \left\{
      \begin{array}{l}
        \text{aksioma} \\
        \text{prielaida iš } \Gamma \\
        \text{prielaida} A \\
        \text{gauta pagal \emph{MP} iš ankstesnių}
      \end{array} \right.
      \]
      Sukonstruokime tokią formulių seką (kuri nebūtinai yra išvedimas):
      \[
      \begin{array}{l l}
        \text{1.} & A \to F_{1}     \\
        \text{2.} & A \to F_{2}     \\
        {}        & \cdots    \\
        \text{i.} & A \to F_{i}     \\
        {}        & \cdots    \\
        \text{k.} & A \to F_{k} = A \to B \\
      \end{array}
      \]
      Dabar kiekvieną $A \to F_{i}$ keičiam tokiu būdu:
      \begin{enumerate}
        \item Jei $F_{i}$ – aksioma, tada keičiam į:
          \[
          \begin{array}{l l l}
            \text{i.1} & F_{i} \to (A \to F_{i}) & \text{1.1 aksioma} \\
            \text{i.2} & F_{1} & \text{aksioma} \\
            \text{i.3} & A \to F_{1} & \text{\emph{MP} iš (i.1) ir (i.2)}\\
          \end{array}
          \]
        \item Jei $F_{i}$ – prielaida iš $\Gamma$, tada keičiam į:
          \[
          \begin{array}{l l l}
            \text{i.1} & F_{i} \to (A \to F_{i}) & \text{1.1 aksioma} \\
            \text{i.2} & F_{1} & \text{prielaida iš } \Gamma\\
            \text{i.3} & A \to F_{1} & \text{\emph{MP} iš (i.1) ir (i.2)}\\
          \end{array}
          \]
        \item Jei $F_{i}$ – prielaida $A$, tada 
          $A \to F_{i} = A \to A$. Keičiam $A \to A$ išvedimu.
        \item Tegu $A \to F_{u}$ yra visos išvestos, kur $u < i$.
          $F_{i}$ buvo gauta iš kažkokių $F_{r}$ ir $F_{l}$ (kur
          $r,l < i$) pritaikius \emph{MP}. Taigi mes jau turime išvestas
          $A \to F_{r}$ ir $A \to F_{l}$. Kadangi $F_{l} = F_{r} \to F_{i}$
          (arba $F_{r} = F_{l} \to F_{i}$), tai:
          \[
          \begin{array}{l l l}
            \text{i.1} & (A \to \overbrace{(F_{r} \to F_{i})}^{F_{l}})%
              \to ((A \to F_{r}) \to (A \to F_{i})) &%
              \text{1.2 aksioma} \\
            \text{i.2} & (A \to F_{r}) \to (A \to F_{i}) &%
              \text{pagal \emph{MP}} \\
            \text{i.3} & A \to F_{i} & \text{pagal \emph{MP}}
          \end{array}
          \]
      \end{enumerate}
      
    \end{description}
  \end{proof}
\end{prop}

Sakysime, jog skaičiavimas yra \emph{pilnas} formulių aibės A atžvilgiu,
jei formulė $\in A$ tada ir tik tada, jei ji yra įrodoma skaičiavimu.

Hilberto tipo teiginių skaičiavimas yra pilnas tapačiai teisingų formulių
aibės atžvilgiu.
