\chapter{Algoritmo samprata}

\begin{defn}[Algoritmas]
  (Intuityvus apibrėžimas.) Veiksmų seka, kuri leidžia spręsti
  vieną ar kitą uždavinį.
\end{defn}

Pagrindinės algoritmo savybės:
\begin{itemize}
  \item žingsnių elementarumas;
  \item diskretumas (algoritmas suskirstytas į atskirus žingsnius);
  \item determinuotumas (atlikus žingsnį aišku, kokį kitą žingsnį 
    reikia atlikti);
  \item masiškumas (skirtas ne vienam uždaviniui, bet jų klasei 
    spręsti).
\end{itemize}

Algoritmo formalizavimo būdai:
\begin{enumerate}
  \item sukurti idealizuotą matematinę mašiną (Turingo, RAM);
  \item apibrėžti rekursyvių funkcijų klasę (Dalinai rekursyvios funkcijos,
    $\lambda$-skaičiavimas).
\end{enumerate}

\begin{prop}
  (Church tezė) Algoritmiškai apskaičiuojamų funkcijų aibė sutampa su 
  rekursyviųjų funkcijų klase. (Šio teiginio neįmanoma įrodyti.)
\end{prop}

\begin{prop}
  Turingo mašinomis apskaičiuojamų funkcijų aibė sutampa su rekursyviųjų
  funkcijų klase.
\end{prop}
