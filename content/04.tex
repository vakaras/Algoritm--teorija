\chapter{Disjunktų dedukcinė sistema. Rezoliucijų metodas}

\begin{defn}[Disjunktas]
  Literų disjunkcija. Litera yra kintamasis arba kintamasis su neigimu.
\end{defn}

\begin{defn}[Atkirtos taisyklė (\emph{AT})]
  \[
  \begin{array}{c l}
    C' \lor p \lor C'' \qquad D' \lor \neg p \lor D'' &%
      \text{, kur $p$ – litera, o $C',C'',D',D''$ – disjunktai.} \\
    \cline{1-1}
    C' \lor C'' \lor D' \lor D'' &
  \end{array}
  \]
\end{defn}

\begin{defn}[$S \vdash C$]
  Sakysime, kad disjunktas $C$ yra išvedamas iš disjunktų aibės $S$, jei
  galima parašyti tokią disjunktų seką, kur kiekvienas disjunktas:
  \begin{itemize}
    \item arba iš aibės $S$,
    \item arba gautas pagal atkirtos taisyklę iš jau parašytų 
  \end{itemize}
  ir kuri (seka) baigiasi disjunktu $C$.
\end{defn}

\begin{defn}[Prieštaringa formulių aibė]
  Sakysime, kad formulių aibė $S$ yra prieštaringa, jei su \emph{bet kokia}
  interpretacija $\nu$ \emph{bent viena} formulė iš aibės $S$ yra 
  klaidinga.
  \begin{note}
    Jei $S$ nėra prieštaringa, tai $S$ – įvykdoma.
  \end{note}
\end{defn}

\begin{prop}
  \label{tg6}
  Jei iš disjunktų aibės $S$ išvedamas disjunktas $C$ ir disjunktas $C$
  nėra įvykdomas (tapačiai klaidingas), tai $S$ – prieštaringa.

  \begin{proof}
    (Prieštaros būdu.)

    Tarkime $S \vdash C$ ir $C$ – nėra įvykdomas, bet aibė $S$ nėra 
    prieštaringa.

    Jei aibė $S$ nėra prieštaringa, tai yra tokia interpretacija $\nu$,
    su kuria visi aibės $S$ disjunktai bus teisingi:
    \[
    \forall D \, (D \in S) : \nu(D) = 1
    \]

    Parodysime, jog visi iš aibės $S$ išvedami disjunktai yra teisingi su 
    ta pačia interpretacija $\nu$.

    Įrodymas matematinės indukcijos metodu, pagal išvedimo ilgį $l$:

    \begin{enumerate}
      \item Jei $l = 1$, tai $D_{1} \in S \implies \nu(D_{1}) = 1$,
        nes visi aibės $S$ disjunktai yra teisingi su interpretacija $\nu$.
      \item Tarkime, kad $\forall i \, (i < m),\: \nu(D_{i}) = 1$.
      \item Panagrinėkime $\nu(D_{m})$:
        \begin{itemize}
          \item Jei $D_{m} \in S$, tai $\nu(D_{m}) = 1$.
          \item Jei $D_{m} \not \in S$, tai $D_{m}$ gautas iš kažkokių 
            $D_{a}$ ir $D_{b}$ pagal atkirtos taisyklę. Pagal matematinės
            indukcijos prielaidą $\nu(D_{a}) = \nu(D_{b}) = 1$, 
            nes $a, b < m$. Kadangi 
            \[ 
            \begin{array}{c l} 
              D_{a} \qquad D_{b} & \text{(AT)} \\ 
              \cline{1-1} 
              D_{m} & {} 
            \end{array} 
            \]
            tai $D_{a} = D'_{a} \lor p$ ir $D_{b} = D'_{b} \lor \neg p$, o
            $D_{m} = D'_{a} \lor D'_{b}$.
            \begin{itemize}
              \item Jei $\nu(p) = 1$, tai $\nu(D'_{b}) = 1$(, nes 
                $\nu(D'_{b} \lor \neg p) = 1$) 
                $\implies \nu(D_{m}) = \nu(D'_{b} \lor D'_{a}) = 1$.
              \item Jei $\nu(\neg p) = 1$, tai $\nu(D'_{a}) = 1$(, nes
                $\nu(D'_{a} \lor p) = 1$)
                $\implies \nu(D_{m}) = \nu(D'_{b} \lor D'_{a}) = 1$.
            \end{itemize}
        \end{itemize}
    \end{enumerate}
    Naudodami matematinę indukciją gavome, kad visi išvedami disjunktai
    yra teisingi su interpretacija $\nu$. Tada ir $\nu(C) = 1 \implies$
    prieštara, nes disjunktas $C$ nėra įvykdomas.
  \end{proof}

\end{prop}

Tuščias disjunktas nėra įvykdomas (yra tapačiai klaidingas).
\begin{notation}
  $\Box$ – tuščias disjunktas.
\end{notation}

Jei iš aibės $S \vdash \Box$, tai aibė $S$ – prieštaringa.

\begin{prop}
  Jei disjunktų aibė $S$ yra prieštaringa, tai iš jos galima išvesti 
  tuščią disjunktą $S \vdash \Box$.

  \begin{proof}
    (Matematinės indukcijos būdu pagal skirtingų kintamųjų kiekį $n$ 
    aibėje $S$.)

    \begin{description}
      \item[Bazė] Jei $n = 1$, tai aibė $S$ gali būti: 
        $\underbrace{\{p\},\: \{\neg p\},}_{\neg \text{prieštaringos}}%
        \:\{p,\neg p\}$.

        $S = \{p, \neg p\}$ – prieštaringa aibė:
        \[
        \begin{array}{l l l}
          1. & p & (\text{iš }S) \\
          2. & \neg p & (\text{iš }S) \\
          3. & \Box & (\text{pagal \emph{AT} iš 1 ir 2})
        \end{array}
        \]
        Jei aibėje $S$ yra vienas kintamasis, tai jei teisinga 
        $S$ – prieštaringa, tai iš jos išvedama $\Box$.

      \item[Prielaida] Tarkime, jei aibėje yra $n < m$ skirtingų 
        kintamųjų, tai jei $S$ – prieštaringa, tai $S \vdash \Box$.

      \item[Indukcinis žingsnis] Tegu disjunktų aibė $S$ turi $n = m$
        skirtingų kintamųjų.

        Padalinkime aibės $S$ disjunktus į 3 aibes (grupes):
        \begin{description}
          \item[$S_{p}$] priklauso tie aibės $S$ disjunktai, kurie neturi
            literos $p$.
          \item[$S_{p}^{+}$] priklauso tie aibės $S$ disjunktai, kurie turi
            literą $p$ (be neigimo).
          \item[$S_{p}^{-}$] priklauso tie aibės $S$ disjunktai, kurie turi
            literą $\neg p$ (su neigimu).
        \end{description}
        Tada: $S = S_{p} \cup S_{p}^{+} \cup S_{p}^{-}$.

        Nagrinėkime aibę $S' = S_{p} \cup \at(S_{p}^{+}, S_{p}^{-})$,
        kur $\at(S_{p}^{+}, S_{p}^{-})$ – yra disjunktų aibė, kuri yra
        gauta pritaikius \emph{AT} visiems disjunktams iš aibių 
        $S_{p}^{+}$ ir $S_{p}^{-}$ pagal kintamąjį $p$. Tegu:
        \[
        \begin{array}{c l l}
          S_{p}^{+} & = \{C_1 \lor p, C_2 \lor p, \dotsc, C_v \lor p \}%
            & \text{ir} \\
          S_{p}^{-} & = \{D_1 \lor \neg p, D_2 \lor \neg p, \dotsc,%
            D_r \lor \neg p \} & {}
        \end{array}
        \]
        tada 
        \begin{align*}
          \at(S_{p}^{+}, S_{p}^{-}) = \{%
          & C_1 \lor D_1, C_1 \lor D_2, \dotsc, C_1 \lor D_r \\
          & C_2 \lor D_1, C_2 \lor D_2, \dotsc, C_2 \lor D_r \\
          & \cdots \\
          & C_v \lor D_1, C_v \lor D_2, \dotsc, C_v \lor D_r \}
        \end{align*}
        Aibėje $S'$ yra $< m$ kintamųjų (nes nėra $p$), todėl aibei 
        $S'$ galioja prielaida: jei $S'$ – prieštaringa 
        $\implies S' \vdash \Box$. Parodysime, jog $S$ ir $S'$ arba
        abi kartu įvykdomos, arba neįvykdomos:
        \begin{enumerate}
          \item Jei $S$ – įvykdoma, tai 
            $\exists \nu : \nu(D) = 1, \forall D \, (D \in S)$. Taip
            pat ir visi disjunktai iš aibės $S_{p}$ yra įvykdomi su 
            ta pačia interpretacija $\nu$, nes $S_{p} \subset S$.
            Taip pat visi disjunktai iš aibės $\at(S_{p}^{+}, S_{p}^{-})$
            yra įvykdomi su interpretacija $\nu$, nes jie yra išvesti iš
            aibės $S$ pritaikius \emph{AT} (žr. \ref{tg6} teiginio 
            įrodymą). Taigi su $\nu$ teisingi visi aibės $S'$ disjunktai,
            tai yra $S'$ – įvykdoma.
          \item Jei $S'$ – įvykdoma, tai yra interpretacija $\nu$, su kuria
            visi aibės $S'$ disjunktai yra teisingi. Kadangi aibėje 
            $S'$ nėra kintamojo $p$, tai interpretacija $\nu$ jam 
            nepriskiria nei 0, nei 1. Papildome interpretaciją $\nu$
            apibrėždami ar kintamasis $p$ yra teisingas ar klaidingas,
            taip kad su interpretacija $\nu$ visi aibės $S$ disjunktai
            būtų teisingi.
            \begin{enumerate}
              \item Jei yra toks $i \, (i=1,2,\dotsc,v)$, su kuriuo 
                $\nu(C_{i}) = 0$, tada apibrėžiame $\nu(p) = 1$, tada
                $\nu(C_1 \lor p) = \nu(C_2 \lor p) = \dots =%
                \nu(C_v \lor p) = 1$

                Kadangi visi aibės $\at(S_{p}^{+}, S_{p}^{-})$ disjunktai 
                yra teisingi su interpretacija $\nu$, tai 
                $\nu(C_i \lor D_1) = \dots = \nu(C_i \lor D_r) = 1$, bet
                $\nu(C_i) = 0$, todėl 
                $\nu(D_1) = \nu(D_2) = \dots = \nu(D_r) = 1 \implies %
                \nu(D_1 \lor \neg p) = \dots = \nu(D_r \lor \neg p) = 1$

                Gavome, kad su $\nu$ visi $S_{p}^{+}, S_{p}^{-}$ ir 
                $S_{p}$ yra teisingi $\implies S$ – įvykdoma.

              \item Jei nėra tokio $i \, (i=1,2,\dotsc,v)$, su kuriuo
                $\nu(C_{i}) = 0$, tada 
                $\nu(C_1) = \nu(C_2) = \dots = \nu(C_v) = 1$, tada ir
                $\nu(C_1 \lor p) = \nu(C_2 \lor p) = \dots =%
                  \nu(C_v \lor p) = 1$.

                Apibrėžkime $\nu(p) = 0$, tada 
                $\nu(D_{1} \lor \neg p) = \nu(D_{2} \lor \neg p) = \dots =%
                \nu(D_{r} \lor \neg p) = 1$.

                Gavome, kad su $\nu$ visi $S_{p}^{+}, S_{p}^{-}$ ir 
                $S_{p}$ yra teisingi $\implies S$ – įvykdoma.

            \end{enumerate}
            Gavome, kad jeigu $S'$ įvykdoma, tai ir $S$ yra įvykdoma.

        \end{enumerate} 
        Aibėje $S'$ yra $< m$ kintamųjų, todėl jei $S'$ prieštaringa,
        tai $S' \vdash \Box$ (prielaida). $S'$ – prieštaringa 
        tada ir tik tada, kai $S$ – prieštaringa, todėl jei 
        $S$ – prieštaringa, tai iš $S$ galima išvesti $\Box$.
    \end{description}

    \emph{Išvada}: $S \vdash \Box \iff S \text{ – prieštaringa}$.

  \end{proof}

\end{prop}

\section{Rezoliucijų metodas}

Jei norime patikrinti ar iš prielaidų $A_1,A_2,\dotsc,A_n$ seka išvada $F$:
\begin{align*}
  A_1,A_2,\dotsc,A_n \vdash F%
  &\iff \text {aibė } B = \{A_1,A_2,\dotsc,A_n,\neg F\}%
    \text{ – prieštaringa} \\
  &\iff \text{disjunktų aibė (gauta paverčiant B aibės elementus į NKF)}\\
  & \qquad  S = \{D_1,D_2,\dotsc,D_m\} \text{ – prieštaringa} \\
  &\iff S \vdash \Box \text{ pagal disjunktų dedukcinę sistemą}.
\end{align*}
