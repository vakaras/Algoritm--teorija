\documentclass{report}

\usepackage{fontspec}
\usepackage{xltxtra}
\usepackage[lithuanian]{babel}
\usepackage{indentfirst}
\usepackage[]{hyperref}
\usepackage[]{amsmath}
\usepackage{amsthm}
\usepackage{amsfonts}

\hypersetup{pdfborder={0 0 0 0}}

\defaultfontfeatures{Mapping=tex-text}

% Apibrėžimai, teiginiai, pastabos.
\swapnumbers

\theoremstyle{plain}
\newtheorem{prop}{Tg.}

\theoremstyle{definition}
\newtheorem{defn}{Ap.}
\newtheorem{exmp}{Pvz.}

\theoremstyle{remark}
\newtheorem*{note}{Pastaba}

\newtheoremstyle{notation}{3pt}{3pt}{}{}{\itshape}{:}{.5em}{}
\theoremstyle{notation}
\newtheorem*{notation}{Žymėjimas}

\title{%
Adomo Birštūno
Algoritmų teorijos paskaitų konspektas}

\author{}

\begin{document}

\maketitle
\bigskip

\tableofcontents

%\chapter{Algoritmo samprata}

\begin{defn}[Algoritmas]
  (Intuityvus apibrėžimas.) Veiksmų seka, kuri leidžia spręsti
  vieną ar kitą uždavinį.
\end{defn}

Pagrindinės algoritmo savybės:
\begin{itemize}
  \item žingsnių elementarumas;
  \item diskretumas (algoritmas suskirstytas į atskirus žingsnius);
  \item determinuotumas (atlikus žingsnį aišku, kokį kitą žingsnį 
    reikia atlikti);
  \item masiškumas (skirtas ne vienam uždaviniui, bet jų klasei 
    spręsti).
\end{itemize}

Algoritmo formalizavimo būdai:
\begin{enumerate}
  \item sukurti idealizuotą matematinę mašiną (Turingo, RAM);
  \item apibrėžti rekursyvių funkcijų klasę (Dalinai rekursyvios funkcijos,
    $\lambda$-skaičiavimas).
\end{enumerate}

\begin{prop}
  (Church tezė) Algoritmiškai apskaičiuojamų funkcijų aibė sutampa su 
  rekursyviųjų funkcijų klase. (Šio teiginio neįmanoma įrodyti.)
\end{prop}

\begin{prop}
  Turingo mašinomis apskaičiuojamų funkcijų aibė sutampa su rekursyviųjų
  funkcijų klase.
\end{prop}
                          % Algoritmo samprata.
\chapter{Hilberto tipo teiginių skaičiavimas}

\begin{defn}[Hilberto tipo teiginių skaičiavimas]
  Skaičiavimas, kuriame yra aksiomų schemos:
  \renewcommand{\theenumii}{\arabic{enumii}}
  \renewcommand{\labelenumii}{\theenumii}
  \begin{enumerate}
    \item%
      \begin{enumerate}
        \item $A \to (B \to A)$
        \item $(A \to (B \to C)) \to ((A \to B) \to (A \to C))$
      \end{enumerate}
    \item%
      \begin{enumerate}
        \item $(A \land B) \to A$
        \item $(A \land B) \to B$
        \item $(A \to B) \to ((A \to C) \to (A \to (B \land C)))$
      \end{enumerate}
    \item%
      \begin{enumerate}
        \item $A \to (A \lor B)$
        \item $B \to (A \lor B)$
        \item $(A \to C) \to ((B \to C) \to ((A \lor B) \to C))$
      \end{enumerate}
    \item%
      \begin{enumerate}
        \item $(A \to B) \to (\neg B \to \neg A)$
        \item $A \to \neg\neg A$
        \item $\neg\neg A \to A$
      \end{enumerate}
  \end{enumerate}
  ir taisyklė \emph{Modus Ponens (MP)}:
  \[
  \begin{array}{r c l}
    \text{prielaidos} \,\{ & A \qquad A \to B & %
      \text{(MP), čia A, B, C – bet kokios teiginių logikos formulės.} \\
    \cline{2-2}
    \text{išvados} \,\{ & B &
  \end{array}
  \]

\end{defn}

Sakysime, jog formulė $F$ yra įvykdoma Hilberto tipo teiginių skaičiavime,
jeigu galima parašyti tokią formulių seką, kurioje kiekviena formulė yra:
\begin{itemize}
  \item arba aksioma,
  \item arba gauta pagal \emph{MP} iš ankstesnių
\end{itemize}
ir kuri (seka) baigiasi formule $F$.

\begin{prop}
  Jei formulė yra įrodoma Hilberto tipo teiginių skaičiavime, tai ji yra
  tapačiai teisinga ir atvirkščiai (jei nėra įrodoma, tai nėra tapačiai
  teisinga).
\end{prop}

Sakysime, jog skaičiavimo aksioma yra \emph{nepriklausoma}, jei ją išmetus
iš skaičiavimo, ji nėra išvedama jame.

Visos Hilberto teiginių skaičiavimo aksiomos yra nepriklausomos.

\section{Dedukcijos teorema}

Sakysime, kad formulė $F$ yra išvedama iš prielaidų 
$A_{1},A_{2},A_{3},\dots,A_{n}$, jei egzistuoja tokia formulių seka, 
kurioje kiekviena formulė yra:
\begin{itemize}
  \item arba aksioma,
  \item arba prielaida,
  \item arba gauta pagal \emph{MP} iš ankstesnių
\end{itemize}
ir kuri (seka) baigiasi formule $F$.
\begin{notation}
  $A_{1},A_{2},\dots,A_{n} \vdash F$
\end{notation}

\begin{prop}
  (Dedukcijos teorema) $\Gamma \vdash A \to B \iff \Gamma,A \vdash B$, 
  kur $A$,$B$ – bet kokios formulės, $\Gamma$ – baigtinė formulių aibė 
  (gali būti tuščia).
  \begin{proof}
    \hfill \\
    \begin{description}
      \item[Būtinumas] \hfill \\
      Jei $\Gamma \vdash A \to B$, tada egzistuoja formulių seka:
      \[
      \begin{array}{l l l}
        \text{1.} & F_{1}   & {}    \\
        \text{2.} & F_{2}   & {}    \\
        {}        & \cdots  & {}    \\
        \text{i.} & F_{i}   & {}    \\
        {}        & \cdots  & {}    \\
        \text{k.} & F_{k} = A \to B & \\
        \text{k+1.} & A & \text{prielaida} \\
        \text{k+2.} & B & \text{\emph{MP} iš k ir (k+1).} \\
      \end{array}
      \]
      kurioje:
      \[
      F_{i} = \left\{
      \begin{array}{l}
        \text{aksioma} \\
        \text{prielaida iš } \Gamma \\
        \text{gauta pagal \emph{MP} iš ankstesnių}
      \end{array} \right.
      \]
      $\Gamma, A \vdash B$.

      \item[Pakankamumas] \hfill \\
      Jei $\Gamma,A \vdash B$, tai yra tokia formulių seka:
      \[
      \begin{array}{l l}
        \text{1.} & F_{1}     \\
        \text{2.} & F_{2}     \\
        {}        & \cdots    \\
        \text{i.} & F_{i}     \\
        {}        & \cdots    \\
        \text{k.} & F_{k} = B \\
      \end{array}
      \]
      kurioje:
      \[
      F_{i} = \left\{
      \begin{array}{l}
        \text{aksioma} \\
        \text{prielaida iš } \Gamma \\
        \text{prielaida} A \\
        \text{gauta pagal \emph{MP} iš ankstesnių}
      \end{array} \right.
      \]
      Sukonstruokime tokią formulių seką (kuri nebūtinai yra išvedimas):
      \[
      \begin{array}{l l}
        \text{1.} & A \to F_{1}     \\
        \text{2.} & A \to F_{2}     \\
        {}        & \cdots    \\
        \text{i.} & A \to F_{i}     \\
        {}        & \cdots    \\
        \text{k.} & A \to F_{k} = A \to B \\
      \end{array}
      \]
      Dabar kiekvieną $A \to F_{i}$ keičiam tokiu būdu:
      \begin{enumerate}
        \item Jei $F_{i}$ – aksioma, tada keičiam į:
          \[
          \begin{array}{l l l}
            \text{i.1} & F_{i} \to (A \to F_{i}) & \text{1.1 aksioma} \\
            \text{i.2} & F_{1} & \text{aksioma} \\
            \text{i.3} & A \to F_{1} & \text{\emph{MP} iš (i.1) ir (i.2)}\\
          \end{array}
          \]
        \item Jei $F_{i}$ – prielaida iš $\Gamma$, tada keičiam į:
          \[
          \begin{array}{l l l}
            \text{i.1} & F_{i} \to (A \to F_{i}) & \text{1.1 aksioma} \\
            \text{i.2} & F_{1} & \text{prielaida iš } \Gamma\\
            \text{i.3} & A \to F_{1} & \text{\emph{MP} iš (i.1) ir (i.2)}\\
          \end{array}
          \]
        \item Jei $F_{i}$ – prielaida $A$, tada 
          $A \to F_{i} = A \to A$. Keičiam $A \to A$ išvedimu.
        \item Tegu $A \to F_{u}$ yra visos išvestos, kur $u < i$.
          $F_{i}$ buvo gauta iš kažkokių $F_{r}$ ir $F_{l}$ (kur
          $r,l < i$) pritaikius \emph{MP}. Taigi mes jau turime išvestas
          $A \to F_{r}$ ir $A \to F_{l}$. Kadangi $F_{l} = F_{r} \to F_{i}$
          (arba $F_{r} = F_{l} \to F_{i}$), tai:
          \[
          \begin{array}{l l l}
            \text{i.1} & (A \to \overbrace{(F_{r} \to F_{i})}^{F_{l}})%
              \to ((A \to F_{r}) \to (A \to F_{i})) &%
              \text{1.2 aksioma} \\
            \text{i.2} & (A \to F_{r}) \to (A \to F_{i}) &%
              \text{pagal \emph{MP}} \\
            \text{i.3} & A \to F_{i} & \text{pagal \emph{MP}}
          \end{array}
          \]
      \end{enumerate}
      
    \end{description}
  \end{proof}
\end{prop}

Sakysime, jog skaičiavimas yra \emph{pilnas} formulių aibės A atžvilgiu,
jei formulė $\in A$ tada ir tik tada, jei ji yra įrodoma skaičiavimu.

Hilberto tipo teiginių skaičiavimas yra pilnas tapačiai teisingų formulių
aibės atžvilgiu.
                          % Hilberto skaičiavimas.
% \chapter{Sekvencinis skaičiavimas}

\begin{defn}[Sekvencija]
  Tokia išraiška $\underbrace{F_{1},F_{2},\dots,F_{n}}_{anticedentas}%
  \vdash \underbrace{G_{1},G_{2},\dots,G_{m}}_{sukcedentas}$, jei 
  \mbox{$n + m > 0$}.
  \begin{note}
    \hfill \\
    \begin{itemize}
      \item $\vdash F$ – formulė yra tapačiai teisinga.
      \item $A_{1},A_{2},\dots,A_{n} \vdash F$ – iš prielaidų seka išvada F.
      \item $A_{1},A_{2},\dots,A_{n} \vdash B_{1},B_{2},\dots,B_{m}$ – iš 
        prielaidų seka bent viena iš išvadų $B_{1},\dots,B_{m}$.
      \item $F \vdash$ – formulė yra tapačiai klaidinga.
    \end{itemize} 
  \end{note}
\end{defn}

\begin{defn}[Sekvencinis skaičiavimas G]
  Skaičiavimas su aksioma $\Gamma',A,\Gamma'' \vdash \Delta',A,\Delta''$
  ir taisyklėmis ($\Gamma,\Gamma',\Gamma'',\Delta,\Delta',\Delta''$ – 
  formulių sekos, $A,B$ – formulės):
  \begin{description}
    \item[$(\vdash \neg)$] 
      \[
      \begin{array}{c}
        \Gamma',A,\Gamma'' \vdash \Delta',\Delta'' \\
        \cline{1-1}
        \Gamma',\Gamma'' \vdash \Delta',\neg A,\Delta''
      \end{array}
      \]
    \item[$(\neg \vdash)$] 
      \[
      \begin{array}{c}
        \Gamma',\Gamma'' \vdash \Delta',A,\Delta'' \\
        \cline{1-1}
        \Gamma',\neg A,\Gamma'' \vdash \Delta',\Delta''
      \end{array}
      \]
    \item[$(\vdash \lor)$] 
      \[
      \begin{array}{c}
        \Gamma',\Gamma'' \vdash \Delta',A,B,\Delta'' \\
        \cline{1-1}
        \Gamma',\Gamma'' \vdash \Delta',A \lor B,\Delta''
      \end{array}
      \]
    \item[$(\lor \vdash)$] 
      \[
      \begin{array}{c}
        \Gamma',A,\Gamma'' \vdash \Delta',\Delta'' \qquad %
        \Gamma',B,\Gamma'' \vdash \Delta',\Delta'' \\
        \cline{1-1}
        \Gamma',A \lor B,\Gamma'' \vdash \Delta',\Delta''
      \end{array}
      \]
    \item[$(\vdash \land)$] 
      \[
      \begin{array}{c}
        \Gamma',\Gamma'' \vdash \Delta',A,\Delta'' \qquad %
        \Gamma',\Gamma'' \vdash \Delta',B,\Delta'' \\
        \cline{1-1}
        \Gamma',\Gamma'' \vdash \Delta',A \land B, \Delta''
      \end{array}
      \]
    \item[$(\land \vdash)$] 
      \[
      \begin{array}{c}
        \Gamma',A,B,\Gamma'' \vdash \Delta',\Delta'' \\
        \cline{1-1}
        \Gamma',A \land B,\Gamma'' \vdash \Delta',\Delta''
      \end{array}
      \]
    \item[$(\vdash \to)$] 
      \[
      \begin{array}{c}
        \Gamma',A,\Gamma'' \vdash \Delta',B,\Delta'' \\
        \cline{1-1}
        \Gamma',\Gamma'' \vdash \Delta',A \to B,\Delta''
      \end{array}
      \]
    \item[$(\to \vdash)$] 
      \[
      \begin{array}{c}
        \Gamma',\Gamma'' \vdash \Delta',A,\Delta'' \qquad %
        \Gamma',B,\Gamma'' \vdash \Delta',\Delta'' \\
        \cline{1-1}
        \Gamma',A \to B,\Gamma'' \vdash \Delta',\Delta''
      \end{array}
      \]
  \end{description}
\end{defn}

Sakysime, kad sekvencija $F_{1},\dots,F_{n} \vdash G_{1},\dots,G_{m}$ yra
išvedame skaičiavime G, jei galime sukonstruoti tokį medį (grafą),
kurio visose viršūnėse yra sekvencijos, šaknyje yra pradinė sekvencija
$F_{1},\dots,F_{n} \vdash G_{1},\dots,G_{m}$ ir kiekvienai viršūnei 
teisinga: jei viršūnės $N$, kurioje 
yra sekvencija S, vaike(-uose) $N_{1}$ (ir $N_{2}$) yra sekvencija(-os)
$S_{1}$ (ir $S_{2}$), tai sekvencija S yra gauta iš sekvencijos(-jų)
$S_{1}$ (ir $S_{2}$) pagal kažkurią taisyklę, ir kurio (medžio)
visuose lapuose yra aksiomos.

Sakysime, kad taisyklė yra \emph{apverčiama}, jei teisinga: jei taisyklės
išvada išvedama, tai išvedamos ir visos taisyklės prielaidos.

Visos sekvencinio skaičiavimo G taisyklės yra apverčiamos.

Jei išvedant gautas lapas, kuriame nėra nei vienos formulės ir jis nėra
aksioma, tai reiškia kad:
\begin{enumerate}
  \item sekvencija yra neišvedama, arba
  \item skaičiavime buvo panaudota neapverčiama taisyklė.
\end{enumerate}

\begin{prop}
  Formulė F yra tapačiai teisinga tada ir tik tada, kai sekvenciniame
  skaičiavime G išvedama sekvencija $\vdash F$.
\end{prop}
                          % Sekvencinis skaičiavimas.
% \chapter{Disjunktų dedukcinė sistema. Rezoliucijų metodas}

\begin{defn}[Disjuktas]
  Literų disjunkcija. Litera yra kintamasis arba kintamasis su neigimu.
\end{defn}

\begin{defn}[Atkirtos taisyklė (\emph{AT})]
  \[
  \begin{array}{c l}
    C' \lor p \lor C'' \qquad D' \lor \neg p \lor D'' &%
      \text{, kur $p$ – litera, o $C',C'',D',D''$ – disjunktai.} \\
    \cline{1-1}
    C' \lor C'' \lor D' \lor D'' &
  \end{array}
  \]
\end{defn}

\begin{defn}[$S \vdash C$]
  Sakysime, kad disjunktas $C$ yra išvedamas iš disjunktų aibės $S$, jei
  galima parašyti tokią disjunktų seką, kur kiekvienas disjunktas:
  \begin{itemize}
    \item arba iš aibės $S$,
    \item arba gautas pagal atkirtos taisyklę iš jau parašytų 
  \end{itemize}
  ir kuri (seka) baigiasi disjunktu $C$.
\end{defn}

\begin{defn}[Prieštaringa formulių aibė]
  Sakysime, kad formulių aibė $S$ yra prieštaringa, jei su \emph{bet kokia}
  interpretacija $\nu$ \emph{bent viena} formulė iš aibės $S$ yra 
  klaidinga.
  \begin{note}
    Jei $S$ nėra prieštaringa, tai $S$ – įvykdoma.
  \end{note}
\end{defn}

\begin{prop}
  \label{tg6}
  Jei iš disjunktų aibės $S$ išvedamas disjunktas $C$ ir disjunktas $C$
  nėra įvykdomas (tapačiai klaidingas), tai $S$ – prieštaringa.

  \begin{proof}
    (Prieštaros būdu.)

    Tarkime $S \vdash C$ ir $C$ – nėra įvykdomas, bet aibė $S$ nėra 
    prieštaringa.

    Jei aibė $S$ nėra prieštaringa, tai yra tokia interpretacija $\nu$,
    su kuria visi aibės $S$ disjunktai bus teisingi:
    \[
    \forall D \, (D \in S) : \nu(D) = 1
    \]

    Parodysime, jog visi iš aibės $S$ išvedami disjunktai yra teisingi su 
    ta pačia interpretacija $\nu$.

    Įrodymas matematinės indukcijos metodu, pagal išvedimo ilgį $l$:

    \begin{enumerate}
      \item Jei $l = 1$, tai $D_{1} \in S \implies \nu(D_{1}) = 1$,
        nes visi aibės $S$ disjunktai yra teisingi su interpretacija $\nu$.
      \item Tarkime, kad $\forall i \, (i < m),\: \nu(D_{i}) = 1$.
      \item Panagrinėkime $\nu(D_{m})$:
        \begin{itemize}
          \item Jei $D_{m} \in S$, tai $\nu(D_{m}) = 1$.
          \item Jei $D_{m} \not \in S$, tai $D_{m}$ gautas iš kažkokių 
            $D_{a}$ ir $D_{b}$ pagal atkirtos taisyklę. Pagal matematinės
            indukcijos prielaidą $\nu(D_{a}) = \nu(D_{b}) = 1$, 
            nes $a, b < m$. Kadangi 
            \[ 
            \begin{array}{c l} 
              D_{a} \qquad D_{b} & \text{(AT)} \\ 
              \cline{1-1} 
              D_{m} & {} 
            \end{array} 
            \]
            tai $D_{a} = D'_{a} \lor p$ ir $D_{b} = D'_{b} \lor \neg p$, o
            $D_{m} = D'_{a} \lor D'_{b}$.
            \begin{itemize}
              \item Jei $\nu(p) = 1$, tai $\nu(D'_{b}) = 1$(, nes 
                $\nu(D'_{b} \lor \neg p) = 1$) 
                $\implies \nu(D_{m}) = \nu(D'_{b} \lor D'_{a}) = 1$.
              \item Jei $\nu(\neg p) = 1$, tai $\nu(D'_{a}) = 1$(, nes
                $\nu(D'_{a} \lor p) = 1$)
                $\implies \nu(D_{m}) = \nu(D'_{b} \lor D'_{a}) = 1$.
            \end{itemize}
        \end{itemize}
    \end{enumerate}
    Naudodami matematinę indukciją gavome, kad visi išvedami disjunktai
    yra teisingi su interpretacija $\nu$. Tada ir $\nu(C) = 1 \implies$
    prieštara, nes disjunktas $C$ nėra įvykdomas.
  \end{proof}

\end{prop}

Tuščias disjunktas nėra įvykdomas (yra tapačiai klaidingas).
\begin{notation}
  $\Box$ – tuščias disjunktas.
\end{notation}

Jei iš aibės $S \vdash \Box$, tai aibė $S$ – prieštaringa.

\begin{prop}
  Jei disjunktų aibė $S$ yra prieštaringa, tai iš jos galima išvesti 
  tuščią disjunktą $S \vdash \Box$.

  \begin{proof}
    (Matematinės indukcijos būdu pagal skirtingų kintamųjų kiekį $n$ 
    aibėje $S$.)

    \begin{description}
      \item[Bazė] Jei $n = 1$, tai aibė $S$ gali būti: 
        $\underbrace{\{p\},\: \{\neg p\},}_{\neg \text{prieštaringos}}%
        \:\{p,\neg p\}$.

        $S = \{p, \neg p\}$ – prieštaringa aibė:
        \[
        \begin{array}{l l l}
          1. & p & (\text{iš }S) \\
          2. & \neg p & (\text{iš }S) \\
          3. & \Box & (\text{pagal \emph{AT} iš 1 ir 2})
        \end{array}
        \]
        Jei aibėje $S$ yra vienas kintamasis, tai jei teisinga 
        $S$ – prieštaringa, tai iš jos išvedama $\Box$.

      \item[Prielaida] Tarkime, jei aibėje yra $n < m$ skirtingų 
        kintamųjų, tai jei $S$ – prieštaringa, tai $S \vdash \Box$.

      \item[Indukcinis žingsnis] Tegu disjunktų aibė $S$ turi $n = m$
        skirtingų kintamųjų.

        Padalinkime aibės $S$ disjunktus į 3 aibes (grupes):
        \begin{description}
          \item[$S_{p}$] priklauso tie aibės $S$ disjunktai, kurie neturi
            literos $p$.
          \item[$S_{p}^{+}$] priklauso tie aibės $S$ disjunktai, kurie turi
            literą $p$ (be neigimo).
          \item[$S_{p}^{-}$] priklauso tie aibės $S$ disjunktai, kurie turi
            literą $\neg p$ (su neigimu).
        \end{description}
        Tada: $S = S_{p} \cup S_{p}^{+} \cup S_{p}^{-}$.

        Nagrinėkime aibę $S' = S_{p} \cup \at(S_{p}^{+}, S_{p}^{-})$,
        kur $\at(S_{p}^{+}, S_{p}^{-})$ – yra disjunktų aibė, kuri yra
        gauta pritaikius \emph{AT} visiems disjunktams iš aibių 
        $S_{p}^{+}$ ir $S_{p}^{-}$ pagal kintamąjį $p$. Tegu:
        \[
        \begin{array}{c l l}
          S_{p}^{+} & = \{C_1 \lor p, C_2 \lor p, \dotsc, C_v \lor p \}%
            & \text{ir} \\
          S_{p}^{-} & = \{D_1 \lor \neg p, D_2 \lor \neg p, \dotsc,%
            D_r \lor \neg p \} & {}
        \end{array}
        \]
        tada 
        \begin{align*}
          \at(S_{p}^{+}, S_{p}^{-}) = \{%
          & C_1 \lor D_1, C_1 \lor D_2, \dotsc, C_1 \lor D_r \\
          & C_2 \lor D_1, C_2 \lor D_2, \dotsc, C_2 \lor D_r \\
          & \cdots \\
          & C_v \lor D_1, C_v \lor D_2, \dotsc, C_v \lor D_r \}
        \end{align*}
        Aibėje $S'$ yra $< m$ kintamųjų (nes nėra $p$), todėl aibei 
        $S'$ galioja prielaida: jei $S'$ – prieštaringa 
        $\implies S' \vdash \Box$. Parodysime, jog $S$ ir $S'$ arba
        abi kartu įvykdomos, arba neįvykdomos:
        \begin{enumerate}
          \item Jei $S$ – įvykdoma, tai 
            $\exists \nu : \nu(D) = 1, \forall D \, (D \in S)$. Taip
            pat ir visi disjunktai iš aibės $S_{p}$ yra įvykdomi su 
            ta pačia interpretacija $\nu$, nes $S_{p} \subset S$.
            Taip pat visi disjunktai iš aibės $\at(S_{p}^{+}, S_{p}^{-})$
            yra įvykdomi su interpretacija $\nu$, nes jie yra išvesti iš
            aibės $S$ pritaikius \emph{AT} (žr. \ref{tg6} teiginio 
            įrodymą). Taigi su $\nu$ teisingi visi aibės $S'$ disjunktai,
            tai yra $S'$ – įvykdoma.
          \item Jei $S'$ – įvykdoma, tai yra interpretacija $\nu$, su kuria
            visi aibės $S'$ disjunktai yra teisingi. Kadangi aibėje 
            $S'$ nėra kintamojo $p$, tai interpretacija $\nu$ jam 
            nepriskiria nei 0, nei 1. Papildome interpretaciją $\nu$
            apibrėždami ar kintamasis $p$ yra teisingas ar klaidingas,
            taip kad su interpretacija $\nu$ visi aibės $S$ disjunktai
            būtų teisingi.
            \begin{enumerate}
              \item Jei yra toks $i \, (i=1,2,\dotsc,v)$, su kuriuo 
                $\nu(C_{i}) = 0$, tada apibrėžiame $\nu(p) = 1$, tada
                $\nu(C_1 \lor p) = \nu(C_2 \lor p) = \dots =%
                \nu(C_v \lor p) = 1$

                Kadangi visi aibės $\at(S_{p}^{+}, S_{p}^{-})$ yra 
                teisingi su interpretacija $\nu$, tai 
                $\nu(C_i \lor D_1) = \dots = \nu(C_i \lor D_r) = 1$, bet
                $\nu(C_i) = 0$, todėl 
                $\nu(D_1) = \nu(D_2) = \dots = \nu(D_r) = 1 \implies %
                \nu(D_1 \lor \neg p) = \dots = \nu(D_r \lor \neg p) = 1$

                Gavome, kad su $\nu$ visi $S_{p}^{+}, S_{p}^{-}$ ir 
                $S_{p}$ yra teisingi $\implies S$ – įvykdoma.

              \item Jei nėra tokio $i \, (i=1,2,\dotsc,v)$, su kuriuo
                $\nu(C_{i}) = 0$, tada 
                $\nu(C_1) = \nu(C_2) = \dots = \nu(C_v) = 1$, tada ir
                $\nu(C_1 \lor p) = \nu(C_2 \lor p) = \dots =%
                  \nu(C_v \lor p) = 1$.

                Apibrėžkime $\nu(p) = 0$, tada 
                $\nu(D_{1} \lor \neg p) = \nu(D_{2} \lor \neg p) = \dots%
                \nu(D_{r}) = 1$.

                Gavome, kad su $\nu$ visi $S_{p}^{+}, S_{p}^{-}$ ir 
                $S_{p}$ yra teisingi $\implies S$ – įvykdoma.

            \end{enumerate}
            Gavome, kad jeigu $S'$ įvykdoma, tai ir $S$ yra įvykdoma.

        \end{enumerate} 
        Aibėje $S'$ yra $< m$ kintamųjų, todėl jei $S'$ prieštaringa,
        tai $S' \vdash \Box$ (prielaida). $S'$ – prieštaringa 
        tada ir tik tada, kai $S$ – prieštaringa, todėl jei 
        $S$ – prieštaringa, tai iš $S$ galima išvesti $\Box$.
    \end{description}

    \emph{Išvada}: $S \vdash \Box \iff S \text{ – prieštaringa}$.

  \end{proof}

\end{prop}

\section{Rezoliucijų metodas}

Jei norime patikrinti ar iš prielaidų $A_1,A_2,\dotsc,A_n$ seka išvada $F$:
\begin{align*}
  A_1,A_2,\dotsc,A_n \vdash F%
  &\iff \text {aibė } B = \{A_1,A_2,\dotsc,A_n,\neg F\}%
    \text{ – prieštaringa} \\
  &\iff \text{disjunktų aibė (gauta paverčiant B aibės elementus į NKF)}\\
  & \qquad  S = \{D_1,D_2,\dotsc,D_m\} \text{ – prieštaringa} \\
  &\iff S \vdash \Box \text{ pagal disjunktų dedukcinę sistemą}.
\end{align*}
                          % Rezoliucijų metodas.
% \chapter{Turingo mašinos ir jų variantai}

\begin{defn}[Determinuota vienajuostė Turingo mašina]
  Ketvertas $<\Sigma,Q,F,\delta>$, kur:
  \begin{description}
    \item[$\Sigma$] – baigtinė aibė – abėcėlė. (Jei nepaminėta kitaip:
      $\Sigma = \{0, 1, b\}$.)
    \item[$Q$] – baigtinė aibė – būsenų aibė. 
      ($Q = \{q_0,q_1,\dotsc,q_n\}$, kur $q_0$ – pradinė būsena.)
    \item[$F$] – galutinių būsenų aibė ($F \subset Q$).
    \item[$\delta$] – perėjimų funkcija. 
      ($\delta: Q \times \Sigma \to Q \times \Sigma \times \{K,N,D\}$)
  \end{description}
\end{defn}

\begin{defn}[Apibrėžta \emph{TM}]
  Sakysime, jog \emph{TM} su pradiniais duomenimis $X$ yra apibrėžta, jei 
  pradžioje į duomenų juostą įrašius žodį $X$, \emph{TM} po baigtinio
  žingsnių kiekio patenka į vieną iš galutinių būsenų.
\end{defn}

\begin{defn}[Turingo mašina apskaičiuoja funkciją]
  Sakysime, kad \emph{TM} $M$ apskaičiuoja funkciją 
  $f(x_1,x_2,\dotsc,x_n)$, jei egzistuoja toks kodavimas abėcėlės 
  $\Sigma$ simboliais $\cod(x_1,x_2,\dotsc,x_n) = \tilde{x}$, kuriam
  teisinga:
  \begin{enumerate}
    \item
      \begin{sloppypar}
        jei $f(x_1,x_2,\dotsc,x_n) = y$ (funkcija apibrėžta), tada
        \emph{TM} $M$ su pradiniais duomenimis $\cod(x_1,x_2,\dotsc,x_n)$
        yra apibrėžta ir baigus darbą, į dešinę nuo skaitymo galvutės yra
        žodis $\cod(y)$;
      \end{sloppypar}
    \item jei $f(x_1,x_2,\dotsc,x_n)$ – neapibrėžta, tada \emph{TM} $M$
      su pradiniais duomenimis $\cod(x_1,x_2,\dotsc,x_n)$ irgi yra
      neapibrėžta.
  \end{enumerate}
\end{defn}

\begin{defn}[Determinuota m-juostė \emph{TM}]
  Ketvertas $<\Sigma,Q,F,\delta>$, kur:
  \begin{description}
    \item[$\Sigma$] – baigtinė aibė – abėcėlė.
    \item[$Q$] – baigtinė būsenų aibė.
    \item[$F$] – galutinių būsenų aibė. ($F \subset Q$)
    \item[$\delta$] – perėjimų funkcija:
      \begin{align*}
        \delta:
        Q &\times \underbrace{\Sigma \times \Sigma \times \cdots
          \times \Sigma}_{m}
        \to\\
        Q &\times \underbrace{%
          \Sigma \times \Sigma \times \cdots \times \Sigma}_{m}%
        \times \underbrace{
          \{K,D,N\} \times \{K,D,N\} \times \cdots \times \{K,D,N\}}_{m}
      \end{align*}
  \end{description}
\end{defn}

\begin{defn}[Nedeterminuota Turingo mašina]
  Turingo mašina, kurios perėjimų funkcija yra daugiareikšmė.
\end{defn}


\section{Baigtiniai automatai}

\begin{defn}[Baigtinis automatas]
  Vienajuostė determinuota Turiningo mašina, kurios perėjimų funkcija
  yra $\delta(q_i, a) = (q_j, a, D)$ ir kuri (\emph{TM}) baigia darbą
  tada, kai pasiekia pirmąją tuščią ląstelę.
\end{defn}

\begin{defn}[Baigtinio automato kalba]
  Aibė žodžių, su kuriais automatas baigia darbą vienoje iš galutinių 
  būsenų.
\end{defn}

\begin{prop}
  Baigtinė aibė yra baigtinio automato kalba.
\end{prop}

\begin{prop}
  Baigtinio automato kalbos $A$ papildinys $\bar{A}$ irgi yra baigtinio
  automato kalba.
\end{prop}

\begin{prop}
  Jei $A_1$ ir $A_2$ yra baigtinio automato kalbos, tai ir 
  $A_1 \cup A_2$, bei $A_1 \cap A_2$ yra baigtinio automato kalbos.

  \begin{proof}
    Grafų $G_1$ ir $G_2$ Dekarto sandauga yra grafas, kurio viršūnės yra 
    visos įmanomos poros $(q_i,q_j)$, kur $q_i$ yra $G_1$ viršūnė, o 
    $q_j$ – $G_2$ viršūnė. Iš viršūnės $(q_i,q_j)$ eis briauna į 
    viršūnę $(q_k,q_l)$ su simboliu $a$, jei grafe $G_1$ eina briauna iš
    viršūnės $q_i$ į viršūnę $q_k$ su simboliu $a$ ir grafe $G_2$ eina
    briauna iš viršūnės $q_j$ į viršūnę $q_l$ su simboliu $a$.
    
    $F_{A \cup B}$ – visos viršūnės $(q_i,q_j)$, kur $q_i \in F_{A}$
    arba $q_j \in F_{B}$.

    $F_{A \cap B}$ – visos viršūnės $(q_i,q_j)$, kur $q_i \in F_{A}$
    ir $q_j \in F_{B}$.

    Baigtinis automatas, kurio perėjimų funkcija vaizduojama grafu
    $G_1 \times G_2$, pradinė būsena yra $(q_0, q_0)$ ir kurio
    galutinių būsenų aibė yra $F_{A \cup B}$, turės kalbą
    $A_1 \cup A_2$. Atitinkamai, baigtinis automatas, kurio galutinių
    būsenų aibė yra $F_{A \cap B}$, turės kalbą $A_1 \cap A_2$.
  \end{proof}
\end{prop}

\begin{prop}
  Baigtinių automatų kalbų
  \begin{enumerate}
    \item konkatenacija $:= \{ uv : u \in A_1, v \in A_2 \}$, kur 
      $A_1$, $A_2$ – automato kalbos;
    \item iteracija $:=\{u_1 u_2 \dots u_k : u_i \in A, i=1,2,\dotsc,k\}$,
      kur $A$ – automato kalba;
    \item atspindys $:=\{a_1 a_2 \dots a_n : a_n a_{n-1} \dots a_1 %
      \in A \}$, kur $A$ – automato kalba
  \end{enumerate}
    irgi yra baigtinių automatų kalbos.
\end{prop}

\section{Algoritmų sudėtingumas}

\begin{notation}
  \hfill \\
  \begin{description}
    \item[$i(v)$] – žodžio $v$ ilgis;
    \item[$t(v)$] – žingsnių kiekis, kurį atlieka \emph{TM}, jei 
      pradinių duomenų juostoje yra žodis $v$.
    \item[$s(v)$] – panaudotų ląstelių kiekis, kurį sunaudojo \emph{TM},
      jei pradinių duomenų juostoje buvo žodis $v$.
  \end{description}
\end{notation}

\begin{defn}[Turingo mašinos M sudėtingumas laiko atžvilgiu]
  Funkcija: $T_{M} (n) = \max \{ t(v) : i(v) = n \}$.
\end{defn}

\begin{defn}[Turingo mašinos M sudėtingumas atminties atžvilgiu]
  Funkcija: $S_{M} (n) = \max \{ s(v) : i(v) = n \}$.
\end{defn}

\begin{defn}[Turingo mašinos kalba]
  Žodžių, su kuriais \emph{TM} baigia darbą galutinėje būsenoje, aibė.
\end{defn}

\begin{defn}[Problema (aibė) išsprendžiama su \emph{TM}]
  Sakysime, jog problema (aibė) A yra išsprendžiama su Turingo mašina,
  jei yra tokia Turingo mašina, kurios kalba yra A.
\end{defn}

Sakysime, kad \emph{TM} sudėtingumas atminties atžvilgiu yra 
$f(n)$, jei $\exists c \,(c >0) : S_{M}(n) = cf(n), \forall n$.

Sakysime, kad \emph{TM} sudėtingumas laiko atžvilgiu yra 
$f(n)$, jei $\exists c \,(c >0) : T_{M}(n) = cf(n), \forall n$.

\begin{defn}[Sudėtingumo klasės]
  \hfill \\
  \begin{itemize}
    \item $DTIME(f(n))$ – sudėtingumo klasė, kuriai priklauso visi
      uždaviniai, kuriems egzistuoja juos sprendžianti daugiajuostė 
      determinuota \emph{TM}, kurios sudėtingumas laiko atžvilgiu 
      yra $f(n)$.
    \item $NTIME(f(n))$ – sudėtingumo klasė, kuriai priklauso visi
      uždaviniai, kuriems egzisutoja juos sprendžianti daugiajuostė 
      nedeterminuota \emph{TM}, kurios sudėtingumas laiko atžvilgiu
      yra $f(n)$.
    \item $DSPACE(f(n))$ – sudėtingumo klasė, kuriai priklauso visi
      uždaviniai, kuriems egzistuoja juos sprendžianti daugiajuostė
      determinuota \emph{TM}, kurios sudėtingumas atminties atžvilgiu
      yra $f(n)$.
    \item $NSPACE(f(n))$ – sudėtingumo klasė, kuriai priklauso visi
      uždaviniai, kuriems egzsituoja juos sprendžianti daugiajuostė
      nedeterminuota \emph{TM}, kurios sudėtingumas atminties atžvilgiu
      yra $f(n)$.
  \end{itemize}
\end{defn}

\begin{defn}[Sudėtingumo klasės]
  \hfill \\
  \begin{itemize}
    \item $L = DSPACE(\log n)$
    \item $NL = NSPACE(\log n)$
    \item $P = DTIME(n^k)$, kur $k$ – konstanta.
    \item $NP = NTIME(n^k)$, kur $k$ – konstanta.
    \item $PSPACE = DSPACE(n^k)$, kur $k$ – konstanta.
    \item $EXP = DTIME(2^{n^{k}})$, kur $k$ – konstanta.
  \end{itemize}
\end{defn}

\begin{prop}
  $L \subseteq NL \subseteq P \subseteq NP %
    \subseteq PSPACE \subseteq EXP$
  ir $NL \not = PSPACE$ ir $P \not = EXP$.
\end{prop}

\section{Porų numeravimas}

\[
\overbrace{\underbrace{(0;0)}_{0}}^{\Sigma = 0},%
\overbrace{\underbrace{(0;1)}_{1},\underbrace{(1;0)}_{2}}^{\Sigma = 1},%
\overbrace{\underbrace{(0;2)}_{3},\underbrace{(1;1)}_{4},
  \underbrace{(2;0)}_{5}}^{\Sigma = 2},
  \underbrace{(0;3)}_{6},\underbrace{(1;2)}_{7},\dotsc
\]

\begin{defn}[Poros numeris \emph{Kantaro} numeracijoje]
  Funkcija $\alpha _{2}(x, y)$. Kairiojo nario funkcija yra 
  $\pi_{2}^{1}(n)$, o dešiniojo $\pi_{2}^{2}(n)$, kur $n$ yra porai 
  priskirtas numeris.
\end{defn}

\begin{prop}
  Poros $(x; y)$ numeris:
  \[
  \alpha _{2} (x, y) = \frac{(x+y)^{2} + 3x + y}{2}
  \]
  \begin{note}
    \[
    \pi^{1}_{2} (n) = n \dotminus \frac{1}{2}%
    \left[\frac{[\sqrt{8n + 1}]+1}{2}\right]%
    \left[ [\sqrt{8n + 1}] - 1 \right]
    \]
  \end{note}
\end{prop}

\begin{defn}[Kantaro funkcijos]
  $\alpha_{n}(x_1,x_2,\dotsc,x_n)$ – rinkinio $x_1,x_2,\dotsc,x_n$ numeris
  Kantaro numeracijoje.
  $\pi^{i}_{n} (K)$ – $K$-ojo rinkinio iš $n$ elementų $i$-asis narys.
\end{defn}

Kantaro funkcijos rekurentinė išraiška:
\[
\begin{cases}
   \alpha _{2} (x,y) = \frac{(x+y)^{2} + 3y + x}{2} \\
   \alpha _{n} (x_1,x_2,\dotsc,x_n) =%
   \alpha _{2} (x_1, \alpha _{n-1} (x_2,x_3,\dotsc,x_n))
\end{cases}
\]

\begin{note} \hfill
\begin{minted}{python}
>>> def alpha(*numbers):
...   if len(numbers) == 2:
...     x, y = numbers
...     return ((x+y)**2 + 3*x + y)/2
...   elif len(numbers) > 2:
...     return alpha(numbers[0], alpha(*numbers[1:]))
...   else:
...     raise Exception('Netinkamas argumentų kiekis!')
>>> alpha(1,0,1,2)
436.0
\end{minted}
\end{note}

\begin{exmp}
  Šis pavyzdys iliustruoja kaip į kiekvieną $n$-argumentų funkciją
  galima žiūrėti kaip į 1-argumento funkciją, kur argumentas yra
  atitinkamo rinkinio numeris. Šiuo atveju $f(x,y)$ yra 2 argumentų
  funkcija, o ją atitinkanti vieno argumento funkcija yra $g(z)$ (iš
  esmės abi jos skaičiuoja tą patį).
  \begin{align*}
    f(x,y) &= 3x + y \\
    g(z) &= 3 \pi^{1}_{2}(z) + \pi^{2}_{2}(z) \\
    f(x,y) &= g(\alpha_{2}(x,y)) = 3x + y
  \end{align*}
\end{exmp}

\section{Baigtinumo problema}

\begin{defn}[Standartinė \emph{TM}]
  Tokia \emph{TM}, kuri:
  \begin{enumerate}
    \item vienajuostė determinuota;
    \item $\Sigma = \left\{ 0, 1, b \right\}$;
    \item kai baigia darbą būdama galutinėje būsenoje, juostoje yra tik
      atsakymas, ir skaitymo galvutė žiūri į pirmąjį iš kairės netuščią
      simbolį;
    \item $|F| = 1$.
  \end{enumerate}
\end{defn}

\begin{prop}
  Kiekvienai \emph{TM} egzistuoja standartinė \emph{TM}, kuri skaičiuoja
  tą pačią funkciją.
\end{prop}

\begin{prop}
  Standartinių \emph{TM} aibė yra skaiti.
  \begin{note}
    Visas standartines \emph{TM} galima sunumeruoti:
    \[
    \begin{array}{r c c c c c l}
     \text{Turingo mašina: } & T_0, & T_1, & T_2, & T_3, & T_4, & \dotsc \\
     \text{Jos skaičiuojama funkcija: } & \varphi_0(x), & \varphi_1(x), &%
       \varphi_2(x), & \varphi_3(x), & \varphi_4(x), & \dotsc
      
    \end{array}
    \]
  \end{note}
\end{prop}

\begin{defn}[Baigtinumo problema]
  Ar yra toks algoritmas, kuris $\mathbb{N}$ skaičių porai $(m,n)$ 
  pasakytų, ar \emph{TM} su numeriu $m$ $(T_m)$ ir pradiniais
  duomenimis $n$ baigia darbą, ar ne.
\end{defn}

\begin{prop}
  Baigtinumo problema neišsprendžiama.
  \begin{proof}
    (Prieštaros būdu.)

    Tarkime, jog egzistuoja toks algoritmas, kuris porai $(m; n)$ pasako,
    ar \emph{TM} $T_{m}$ su duomenimis $n$ yra apibrėžta, ar ne. Tada
    egzistuoja tokia algoritmiškai apskaičiuojama funkcija:
    \[
    g(\alpha_{2}(x,y)) =%
    \begin{cases}
      1, & \text{jei } \varphi_{x}(y) < \infty \text{ (apibrėžta)}, \\
      0, & \text{jei } \varphi_{x}(y) = \infty \text{ (neapibrėžta)}.
    \end{cases}
    \]
    Tada yra \emph{TM}, kuri apskaičiuoja funkciją $g(z)$.
    Tada yra \emph{TM}, kuri apskaičiuoja funkciją:
    \[
    f(x) =%
    \begin{cases}
      1, & \text{ jei } g(\alpha_{2}(x,x)) = 0, \\
      \infty, & \text{ jei } g(\alpha_{2}(x,x)) = 1.
    \end{cases}
    \]
    \emph{TM} skaičiuojančią $g(z)$ pažymėkime $M_{1}$.
    \begin{align*}
      M_{1}: & \: Q_{1} \text{ – būsenų aibė;} \\
      & \: F_{1} \text{ – galutinių būsenų aibė.}
    \end{align*}
    Sukonstruokime \emph{TM} $M_{2}$ skaičiuojančią $f(x)$:

    Kai $M_{1}$ baigia darbą, tai galimi du variantai:
    \begin{enumerate}
      \item \emph{TM} yra galutinėje būsenoje $q_{F} \in F_{1}$ ir 
        juostoje yra vienintelis simbolis $1$, į kurį ir žiūri 
        skaitymo galvutė.
      \item \emph{TM} yra galutinėje būsenoje $q_{F} \in F_{1}$ ir 
        juostoje yra vienintelis simbolis $0$, į kurį ir žiūri 
        skaitymo galvutė.
    \end{enumerate}
    \emph{TM} $M_{2}$ tokia pati, kaip ir $M_{1}$, tik 
    $Q_{2} = Q_{1} \cup \{ q^{*} \}$, $F_{2} = \{ q^{*} \}$, 
    $q_{F}$ – nėra galutinė būsena ir:
    \begin{align*}
      \delta(q_{F},0) & = (q^{*}, 1, N) \to f(x)\text{ – apibrėžta;}\\
      \delta(q_{F},1) & = (q_{F}, 1, N) \to f(x)\text{ – neapibrėžta.}
    \end{align*}

    Kadangi yra \emph{TM} ($M_{2}$), kuri apskaičiuoja funkciją $f(x)$,
    tai yra toks numeris $l$, kad $T_{l} = M_{2}$. $T_{l}$ apskaičiuoja
    $\varphi_{l}(x)$, todėl $f(x) = \varphi_{l}(x)$. Panagrinėkime 
    $\varphi_{l}(l)$:
    \begin{enumerate}
      \item Jei $\varphi_{l}(l) < \infty$, tai 
        $g(\alpha_{2}(l,l)) = 1 \implies f(l) = \infty \implies%
        \varphi_{l}(l) = \infty$;
      \item Jei $\varphi_{l}(l) = \infty$, tai 
        $g(\alpha_{2}(l,l)) = 0 \implies f(l) = 1 \implies%
        \varphi_{l}(l) < \infty$.
    \end{enumerate}
    Gauname prieštarą, todėl neegzistuoja toks algoritmas, kuris 
    porai $(m;n)$ pasakytų, ar \emph{TM} $T_{m}$ su pradiniais duomenis
    $n$ baigia darbą, ar ne.
  \end{proof}
\end{prop}

\begin{defn}[Aibės charakteringoji funkcija]
  Aibės A charakteringoji funkcija yra:
  \[
  \chi _{A}(x) =%
  \begin{cases}
    1, & \text{ jei } x \in A \\
    0, & \text{ jei } x \not \in A
  \end{cases}
  \]
\end{defn}

\begin{defn}[Rekursyvi aibė]
  Sakysime, kad aibė yra rekursyvi, jei jos charakteringoji funkcija yra
  visur apibrėžta rekursyvi funkcija.
\end{defn}

\begin{note}
  (Rice teorema.) Jei aibė $X$ yra vieno argumento dalinai rekursyvių
  funkcijų aibė ir nesutampa nei su $\emptyset$, nei su visa vieno
  argumento dalinai rekursyvių funkcijų aibe (yra poaibis), tai 
  aibė $A = \left\{ x : \varphi_{x} \in X \right\}$ nėra rekursyvi.
  \begin{exmp}
    $X$ – funkcijų, kurios visada grąžina $1$ aibė: $X = \{ f(x) = 1 \}$.
    Tada $A = \{ x : \varphi_{x} \in X \}$ nėra rekursyvi $\implies$
    nėra algoritmo, kuris pasakytu ar $x \in A$.
  \end{exmp}
\end{note}

                          % Turingo mašinos.
% \chapter{$\lambda$-skaičiavimas}

\begin{defn}[$\lambda$–skaičiavimo termas]
  \hfill \\
  \begin{itemize}
    \item Jei $u$ – kintamasis, tai $u$ yra termas.
    \item Jei $E_{1}$ ir $E_{2}$ yra termai, tai ir $(E_{1}E_{2})$ yra 
      termas.
    \item Jei $E$ yra termas, o $x$ – kintamasis, tai ir 
      $\lambda x.E$ irgi yra termas.
  \end{itemize}
  \begin{note}
    Kintamuosius žymime mažosiomis raidėmis.
  \end{note}
  \begin{note}
    Skliaustai būtini: $(xy)z \not = x(yz)$!
  \end{note}
\end{defn}

Kintamojo įeitis terme yra jo pasitaikymas jame.
\begin{exmp}
  Jei turime termą:
  \[
  (\lambda x.((yz)(\lambda z.((zx)y)))(x(\lambda y.(xy)))),
  \]
  tai:
  \[
  (\lambda \underbrace{x}_{1}.((yz)
  (\lambda z.((z\underbrace{x}_{2})y)))
  (\underbrace{x}_{3}(\lambda y.(\underbrace{x}_{4}y)))),
  \]
  sunumeruotos yra kintamojo $x$ įeitys.
\end{exmp}

Kintamojo įeitis $x$ yra suvaržyta, jei patenka į $\lambda x.$ veikimo
sritį. Kitaip $x$ įeitis yra laisva.
\begin{note}
  $\lambda x.E$ terme $\lambda x.$ veikimo sritis yra termas $E$.
\end{note}
\begin{note}
  $\lambda x.xu = (\lambda x.x)u \not = \lambda x.(xu)$ 
\end{note}

Termai $E_{1}$ ir $E_{2}$ yra $\alpha$-ekvivalentūs, jei $E_{2}$ yra
gautas iš $E_{1}$, jame visas laisvas kažkurio kintamojo $x$ įeitis
pakeitus nauju kintamuoju.

Termas $\lambda x.E_{1}$ yra $\alpha$-ekvivalentus termui 
$\lambda y.E_{2}$, jei termas $E_{2}$ yra gautas iš termo $E_{1}$, jame
visas  laisvas $x$ įeitis pakeitus nauju kintamuoju $y$.

\begin{defn}[Redeksas ir jo santrauka]
  Termas pavidalo $(\lambda x.E)Y$, kur $E$ ir $Y$ yr(a termai, vadinamas
  redeksu. Redekso $(\lambda x.E)Y$ santrauka yra termas $E[^Y/_x]$
  – termas $E$, kuriame visos laisvos kintamojo $x$ įeitys yra pakeistos
  termu $Y$.
  \begin{exmp}
    Redekso $(\lambda x.\underbrace{(ux))}_{E}\underbrace{(zz)}_{Y}$
    santrauka yra termas $u(zz)$.
  \end{exmp}
  \begin{exmp}
    Redekso $(\lambda x.((ux)(\lambda x.(xy))))(zz)$ santrauka yra 
    termas $(u(zz))(\lambda x.(xy))$.
  \end{exmp}
\end{defn}

\begin{defn}[$\beta$-redukcija]
  $\lambda$-skaičiavimo termo $E$ $\beta$-redukcija vadinama termų seka
  $E_1 \triangleright E_2 \triangleright E_3 \triangleright %
  E_4 \triangleright \cdots \triangleright E_k \triangleright$, kur
  $E_1 = E$, ir $E_{i+1}$ yra gautas iš $E_{i}$ jame pirmąjį iš kairės 
  redeksą pakeitus jo santrauka.
\end{defn}

\begin{defn}[Normalinis termas]
  Termas, kuriame nėra redeksų. Termas yra nenormalizuojamas, jeigu jo 
  $\beta$-redukcija yra begalinė.
\end{defn}

\begin{defn}
  $\lambda$-skaičiavimo loginės konstantos yra 
  $\underbrace{\lambda x.\lambda y.y}_{klaidinga}$ ir 
  $\underbrace{\lambda x.\lambda y.x}_{teisinga}$.
  \begin{notation}
    $\lambda x.\lambda y.y = 0$ ir $\lambda x.\lambda y.x = 1$.
  \end{notation}
\end{defn}

\begin{defn}
  $\lambda$-skaičiavimo natūrinis skaičius $k$ yra 
  $\lambda f.\lambda x.(f^{k} x)$, kur
  $f^{k} x = \underbrace{f(f(\dots f(f(}_{\text{k kartų}} x ))\dots))$.
  \begin{notation}
    Skaičius 2 žymimas 
    $\underline{2} = \lambda f.\lambda x(f^{2} x) =%
    \lambda f.\lambda x(f(fx))$
  \end{notation}
\end{defn}

\begin{exmp}
  $(x1)\underline{1} = (x(\lambda x.\lambda y.x))(\lambda f.\lambda x.(fx))$
\end{exmp}

\begin{defn}
  Sakysime, kad termas $E$ apibrėžia dalinę funkciją 
  $f(x_1,x_2,\dotsc,x_n)$, jei:
  \begin{itemize}
    \item jei $f(K_1,K_2,\dotsc,K_n) = K$, tai termas 
      $(\dots((EK_1)K_2)\dots)K_n$ redukuojamas ($\beta$-redukcijoje) į
      termą $K$;
    \item jei $f(K_1,K_2,\dotsc,K_n) = \infty$, tai termas 
      $(\dots((EK_1)K_2)\dots)K_n$ yra neredukuojamas.
  \end{itemize}
\end{defn}

\begin{prop}
  Kiekvienai algoritmiškai apskaičiuojamai funkcijai egzistuoja ją 
  apibrėžiantis termas.
\end{prop}


                          % Baigtiniai automatai.
% \chapter{Primityviai rekursyvios funkcijos}

Kompozicijos operatorius: sakysime, kad funkcija
$f(x_1,x_2,\dotsc,x_n)$ yra gauta iš funkcijų 
$g_{i}(x_1,x_2,\dotsc,x_n)\,(i=1,2,3,\dotsc,m)$ ir funkcijos
$h(x_1,x_2,\dotsc,x_m)$, jei 
$f(x_1,x_2,\dotsc,x_n) = %
h(g_{1}(x_1,\dotsc,x_n),\dotsc,g_{m}(x_1,\dotsc,x_n))$.

\begin{defn}
  Sakysime, jog funkcija $f(x_1,x_2,\dotsc,x_n)$ yra gauta pagal 
  primityviosios rekursijos operatorių iš 
  $g(x_1,x_2,\dotsc,x_{n-1})$ ir 
  $h(x_1,x_2,\dotsc,x_n,x_{n+1})$, jei teisinga:
  \begin{align*}
    f(x_1,x_2,\dotsc,x_{n-1},0) &= g(x_1,x_2,\dotsc,x_{n-1}) \text{ ir} \\
    f(x_1,x_2,\dotsc,x_{n-1},y+1) &= %
      h(x_1,x_2,\dotsc,x_{n-1},y,f(x_1,x_2,\dotsc,x_{n-1},y))
  \end{align*}
\end{defn}

\begin{defn}[Primityviai rekursyvių funkcijų aibė (\emph{PR})]
  Primityviai rekursyvių funkcijų aibė sutampa su aibe, kuriai
  priklauso bazinės funkcijos: 
  $0, s(x) = x+1, pr_{m}^{i}(x_1,x_2,\dotsc,x_m) = x_{i}$ ir
  kuri yra uždara kompozicijos ir primityviosios rekursijos
  operatorių atžvilgiu.
\end{defn}

Visos \emph{PR} funkcijos yra apibrėžtos su visais natūraliaisiais 
skaičiais.

Sakysime, kad $\mathbb{N}$ skaičių poaibis yra primityviai rekursyvus,
jei jo charakteringoji funkcija priklauso \emph{PR}.

\begin{prop}
  Kantaro funkcijos $\alpha_{n}(x_1,x_2,\dotsc,x_n)$ ir 
  $\pi^{i}_{n}(K)$ yra primityviai rekursyvios.
\end{prop}

\begin{prop}
  Jei funkcija $g(x_1,x_2,\dotsc,x_n) \in \PR$, tai ir 
  funkcija
  $f(x_1,x_2,\dotsc,x_n) = \sum _{i=0} ^{x_n} g(x_1,x_2,\dotsc,x_n)%
  \in \PR$
\end{prop}

\begin{prop}
  Jei $f_{i}(x_1,x_2,\dotsc,x_n) \in \PR$ ir 
  $\alpha_{j}(x_1,x_2,\dotsc,x_n) \in \PR$ 
  $(i=1,2,\dotsc,s,s+1;\,j=1,2,\dotsc,s)$, tada ir funkcija:
  \[
  h(x_1,x_2,\dotsc,x_n) =%
  \begin{cases}
    f_{1}(x_1,x_2,\dotsc,x_n), 
      &\text{ jei } \alpha_{1}(x_1,x_2,\dotsc,x_n)=0,\\
    f_{2}(x_1,x_2,\dotsc,x_n), 
      &\text{ jei } \alpha_{2}(x_1,x_2,\dotsc,x_n)=0,\\
    \cdots & {} \\
    f_{s}(x_1,x_2,\dotsc,x_n), 
      &\text{ jei } \alpha_{s}(x_1,x_2,\dotsc,x_n)=0,\\
    f_{s+1}(x_1,x_2,\dotsc,x_n), 
      &\text{ kitu atveju }
  \end{cases}
  \]
  yra primityviai rekursyvi.
  \begin{note}
    Su bet kuriuo rinkiniu $(x_1,x_2,\dotsc,x_n)$ ne daugiau nei viena
    $\alpha_{i}(x_1,x_2,\dotsc,x_n)$ ($i=1,2,\dotsc,s$) gali būti lygi 0.
  \end{note}
\end{prop}

\begin{defn}[Iteracijos operatorius]
  Sakysime, kad funkcija $f(x)$ yra gauta pagal iteracijos operatorių iš
  funkcijos $g(x)$, jei teisinga:
  \[
  \begin{cases}
    f(0) &= 0 \\
    f(y+1) &= g(f(y))
  \end{cases}
  \]
\end{defn}

\begin{prop}
  \label{pr1arg}
  Visų vieno argumento $\PR$ funkcijų aibė sutampa su aibe, kuriai
  priklauso funkcijos $s(x) = x+1$, 
  $q(x) = x \dotminus \left[ \sqrt{x} \right]^2$ ir kuri yra 
  uždara kompozicijos, sudėties ir iteracijos operatorių atžvilgiu.
\end{prop}

\section{Dalinai rekursyvios funkcijos}

\begin{defn}[Minimizacijos operatorius]
  Sakysime, kad funkcija $f(x_1,x_2,\dotsc,x_n)$ gauta pagal minimizacijos
  operatorių iš funkcijos $g(x_1,x_2,\dotsc,x_n)$, jei:
  \[
  f(x_1,x_2,\dotsc,x_n) = \mu_{y}(g(x_1,x_2,\dotsc,x_{n-1},y) = x_n),
  \]
  tai yra pati mažiausia natūrali $y$ reikšmė, kuriai teisinga
  $g(x_1,x_2,\dotsc,x_{n-1},y) = x_n$.

  \begin{exmp}
    $f(x,y) = \mu_{z}((s(z)\dotminus x)\cdot \sg(y \dotminus z)=x)$
  \end{exmp}
\end{defn}

\begin{defn}[Dalinai rekursyvių funkcijų aibė (DR)]
  Dalinai rekursyvių funkcijų aibė sutampa su aibe, kuriai priklauso 
  bazinės funkcijos $0, s(x) = x+1, pr_{n}^{i}(x_1,x_2,\dotsc,x_n) = x_{i}$
  ir kuri yra uždara kompozicijos, primityviosios rekursijos ir
  minimizacijos operatorių atžvilgiu.

  \begin{note}
    Jei $f(x_1,x_2,\dotsc,x_n) = \mu_{y}(g(x_1,x_2,\dotsc,x_{n-1},y)=x_n)$,
    tai $f(x_1,x_2,\dotsc,x_n)$ reikšmė apskaičiuojama remiantis tokiu
    algoritmu:
    \[
    \verb|for (y = 0; |g(x_1,x_2,\dotsc,x_{n-1},y) %
      \neq x_n\verb|; y++);|      
    \]
    \[
    f(x_1,x_2,\dotsc,x_n) =%
    \begin{cases}
      y, & \text{jei pavyko rasti $y$ arba} \\
      \infty, & \text{jei $y$ neegzistuoja, arba skaičiuojant buvo %
        gauta neapibrėžtis.}
    \end{cases}
    \]
  \end{note}
\end{defn}

\begin{defn}[Bendrųjų rekursyviųjų funkcijų aibė ($\BR$)]
  Aibė sudaryta iš visų $\DR$ funkcijų, kurios yra apibrėžtos su visais
  natūraliaisiais argumentais.
\end{defn}

\begin{prop}
  $\PR \subseteq \BR \subseteq \DR$
\end{prop}

\begin{note}
  Žinomos primityviai rekursyvios funkcijos:
  \begin{itemize}
    \item bazinės funkcijos:
      \begin{align*}
        & 0,\\
        & s(x) = x + 1, \\
        & pr^{i}_{n}(x_1,x_2,\dotsc,x_n) = x_i;
      \end{align*}
    \item vieno argumento $\PR$ funkcijų bazinės funkcijos:
      \begin{align*}
        s(x) &= x + 1, \\
        q(x) &= x \dotminus \left[ \sqrt{x} \right]^{2};
      \end{align*}
    \item kitos žinomos $\PR$ funkcijos:
      \begin{align*}
        \sg(x) &= %
        \begin{cases}
          1, & \text{ jei } x > 0 \\
          0, & \text{ jei } x = 0
        \end{cases}, \\
        \sgi(x) &= %
        \begin{cases}
          0, & \text{ jei } x > 0 \\
          1, & \text{ jei } x = 0
        \end{cases}, \\
        x \dotminus y &=%
        \begin{cases}
          x - y, & \text{ jei } x > y \\
          0, & \text{ kit atveju}
        \end{cases}, \\
        x + y,& \\
        x \cdot y, & \\
        |x - y|, & \\
        [x / y]. & 
      \end{align*}
  \end{itemize}
\end{note}

\begin{note}
  Žinomos dalinai rekursyvios funkcijos:
  \begin{itemize}
    \item dalinis skirtumas:
      \[
      x - y =%
      \begin{cases}
        x - y, & \text{ jei } x \geq y \\
        \infty, & \text{ kitu atveju }
      \end{cases},
      \]
    \item dalinė dalyba:
      \[
      x / y =%
      \begin{cases}
        x / y, & \text{ jei $x$ dalosi iš $y$ }\\
        \infty, & \text{ kitu atveju }
      \end{cases},
      \]
    \item dalinė šaknis:
      \[
      \sqrt{x} =%
      \begin{cases}
        \sqrt{x}, & \text{ jei $x$ yra natūralaus skaičiaus kvadratas} \\
        \infty, & \text{ kitu atveju }
      \end{cases}.
      \]
  \end{itemize}
\end{note}

\section{Rekursyviai skaičiosios aibės}

\begin{defn}[Rekursyviai skaiti aibė]
  \label{rsadr}
  Aibė $A$ yra rekursyviai skaiti, jei sutampa su kažkurios $\DR$ funkcijos
  apibrėžimo sritimi.
\end{defn}

\begin{defn}[Rekursyviai skaiti aibė]
  \label{rsapr}
  Netuščia aibė $A$ yra rekursyviai skaiti, jei sutampa su kažkurios
  $\PR$ funkcijos reikšmių sritimi.
\end{defn}

\begin{defn}[Rekursyviai skaiti aibė]
  \label{rsalg}
  Aibė $A$ yra rekursyviai skaiti, jei $\exists$ tokia $\PR$ funkcija
  $f(a,x)$, kad lygtis $f(a, x) = 0$ turi sprendinį tada ir tik tada, kai
  $a \in A$.
\end{defn}

\begin{prop}
  Visi trys (\ref{rsadr}, \ref{rsapr} ir \ref{rsalg}) rekursyvios aibės
  apibrėžimai yra ekvivalentūs netuščios aibės atžvilgiu.
  \begin{proof}
    (Dalinis \ref{rsapr} $\implies$ \ref{rsadr}, 
    \ref{rsapr} $\implies$ \ref{rsalg} ir 
    \ref{rsalg} $\implies$ \ref{rsapr})

    \begin{description}
      \item[(\ref{rsapr} $\implies$ \ref{rsadr})]
        Tarkime, kad aibė $A$ yra rekursyviai skaiti, nes $\exists \PR$
        funkcija $h(x)$ tokia, kad $A$ sutampa su $h$ reikšmių 
        sritimi (\ref{rsapr} ap.):
        \[
        A = \left\{ h(0), h(1), h(2), h(3), \dotsc \right\}.
        \]
        Imkime funkciją $f(x) = \mu_{y}(h(y) = x)$. Ji yra dalinai 
        rekursyvi, nes yra gauta pagal minimizacijos operatorių iš
        dalinai rekursyvios funkcijos $h (\in \PR \subseteq \DR)$.
        \begin{itemize}
          \item Jei $a \in A$, tai $\exists i(i \in \mathbb{N})$, kad
            $h(i) = a \implies f(a) = \mu_{y}(h(y) = a) < \infty$ 
            – apibrėžta.
          \item Jei $a \not \in A$, tai 
            $\forall j (j \in \mathbb{N}): \, h(j) \neq a \implies%
            f(a) = \mu_{y}(h(y) = a) = \infty$ – neapibrėžta.
        \end{itemize}
        Aibė $A$ sutampa su funkcijos $f(x)$ apibrėžimo sritimi ir
        $f(x) \in \DR \implies$ \ref{rsadr} ap.
      \item[(\ref{rsapr} $\implies$ \ref{rsalg})] 
        Tarkime, jog aibė $A$ sutampa su $\PR$ funkcijos
        $h(x)$ apibrėžimo sritimi (\ref{rsapr} ap.). Tai yra:
        \[
        A = \left\{ h(0),h(1),h(2),h(3),\dotsc \right\}.
        \]
        Imkime funkciją $f(a,x) = |h(x)-a|$. $f(x) \in \PR$, nes gauta
        iš $\PR$ funkcijų pritaikius kompozicijos operatorių. Lygtis
        $f(a,x) = 0$ yra $|h(x) - a| = 0$.
        \begin{itemize}
          \item Jei $a \in A$, tai $\exists$ toks $i$, kad 
            $h(i)=a \implies |h(x) - a| = 0$, turi sprendinį $x = i$.
          \item Jei $a \not \in A$, tai 
            $\forall j (j \in \mathbb{N}): h(j) \neq a \implies %
            |h(x) - a| > 0$ – lygtis neturi sprendinių.
        \end{itemize}
        Lygtis $f(a,x)=0$, turi sprendinį tada ir tik tada, 
        kai $a \in A \implies$ \ref{rsalg} ap.
      \item[(\ref{rsalg} $\implies$ \ref{rsapr})]
        Tarkime, jog $a \in A$ tada ir tik tada, kai lygtis 
        $f(a,x) = 0$, kur $f(a,x) \in \PR$, turi sprendinį.
        Imkime funkciją
        \[
        h(t)=\pi^{1}_{2}(t) \cdot \sgi(f(\pi^{1}_{2}(t),\pi^{2}_{2}(t)))%
        + d \cdot \sg(f(\pi^{1}_{2}(t), \pi_{2}^{2}(t))), 
        \]
        kur $d$ – bet koks aibės $A$ elementas. $h(t) \in \PR$, nes gauta
        iš žinomų $\PR$ funkcijų pritaikius kompozicijos operatorių.
        Parodysime, kad $h(t)$ reikšmių sritis sutampa su aibe $A$:
        \begin{itemize}
          \item Jei $a \in A$, tai lygtis $f(a,x) = 0$ turi sprendinį
            $x = u$. Pažymėkime $t = \alpha_{2}(a, u)$. Tada
            \begin{align*}
              h(t) &= h(\alpha_{2}(a, u)) \\
              &= a \cdot \sgi(\underbrace{f(a, u)}_{= 0}) + %
                d \cdot \sg(\underbrace{f(a, u)}_{= 0}) \\
              &= a \cdot \sgi(0) + d \cdot \sg(0) \\
              &= a
            \end{align*}
          \item Jei $t = \alpha_{2}(a, v)$, kad $f(a, v) \neq 0$, tai
            \begin{align*}
              h(t) &= h(\alpha_{2}(a, v)) \\
              &= a \cdot \sgi(\underbrace{f(a, v)}_{> 0}) + %
                d \cdot \sg(\underbrace{f(a, v)}_{> 0}) \\
              &= a \cdot 0 + d \cdot 1 \\
              &= d
            \end{align*}
        \end{itemize}
        Gavome, kad $h(t)$ su bet kokiu $t$ įgyja reikšmę $\in A$ ir 
        $h(t)$ įgyja visas reikšmes iš aibės $A$. Kadangi 
        $h(t)$ reikšmių sritis yra aibė $A$ ir $h(t) \in \PR$, tai
        gavome \ref{rsapr} apibrėžimą.
    \end{description}
  \end{proof}
\end{prop}

\begin{note}
  Rekursyvių ir rekursyviai skaičių aibių skirtumas:
  \begin{itemize}
    \item $A$ – rekursyvi, jei 
      \[
      \exists \, \chi_{A}(a) =%
      \begin{cases}
        1, & \text{ jei } a \in A \\
        0, & \text{ jei } a \not \in A
      \end{cases}%
      \in \BR
      \]
      (Funkcija visur apibrėžta.)
    \item $A$ – rekursyviai skaiti, jei
      \[
      \exists \, \chi_{A}(a) =%
      \begin{cases}
        1, & \text{ jei } a \in A \\
        \infty, & \text{ jei } a \not \in A
      \end{cases}%
      \in \DR
      \]
      (Funkcija ne visur apibrėžta.)
  \end{itemize}
\end{note}

Rekursyviai skaičių aibių savybės:
\begin{enumerate}
  \item Rekursyvi aibė $A$ yra ir rekursyviai skaiti:
    \[
    \exists \: \chi_{A}(a) =%
    \begin{cases}
      1, & \text{ jei } a \in A \\
      0, & \text{ jei } a \not \in A
    \end{cases},
    \]
    tada aibė $A$ sutampa su $\DR$ funkcijos $f(a) = \chi_{A}(a) - 1$
    apibrėžimo sritimi. Pagal \ref{rsadr} apibrėžimą $A$ yra 
    rekursyviai skaiti.
  \item Baigtinė aibė yra ir rekursyvi ir rekursyviai skaiti:
    \[
    A = \left\{ a_1, a_2, \dotsc, a_n \right\} \text{ – baigtinė aibė.}
    \]
    Tada, jos charakteringoji funkcija:
    \[
    \chi_{A}(a) = \sg(\sgi(|a - a_1|) + \sgi(|a - a_2|) + \dotsc +%
      \sgi(|a - a_n|)).
    \]
\end{enumerate}

\begin{prop}
  Jeigu aibė $A$ yra rekursyviai skaiti, bet $A$ nėra rekursyvi, tai
  jos papildinys $\overline{A}$ nėra nei rekursyvi, nei rekursyviai
  skaiti aibė.
  \begin{proof}
    \hfill \\
    \begin{enumerate}
      \item Tarkime, kad $\overline{A}$ yra rekursyviai skaiti.
      \item Tada pagal \ref{rsapr} apibrėžimą yra $\PR$ funkcija
        $g(x)$, tokia kad $\overline{A} = \{g(0), g(1), g(2), \dotsc \}$
      \item Kadangi aibė $A$ yra rekursyviai skaiti (duota), tai 
        pagal \ref{rsapr} apibrėžimą yra $\PR$ funkcija $f(x)$ tokia, kad
        $A = \{ f(0), f(1), f(2), \dotsc \}$.
      \item Panagrinėkime funkciją 
        $h(x) = \mu_{z}(|f(z)-x| \cdot |g(z)-x|=0)$. $h(x)$ yra
        $\BR$, nes gauta iš $\PR$ funkcijų taikant kompozicijos, bei
        minimizacijos operatorius ir yra apibrėžta su visais 
        $x \, (x \in \mathbb{N})$, nes $A \cup \overline{A} = \mathbb{N}$.
        \begin{itemize}
          \item jei $x \in A$, tada 
            $\exists z: f(z) = x \implies |f(z) - x| = 0$,
          \item jei $x \not \in A \implies x \in \overline{A}$, tada
            $\exists z: g(z) = x \implies |g(z) - x| = 0$.
        \end{itemize}
      \item 
        \begin{itemize}
          \item Jei $x \in A$, tada $h(x) = z$, kad $f(z) = x$.
          \item Jei $x \in \overline{A}$, tada $h(x) = z$, kad 
            $g(z) = x$ ir $f(z) \neq x$.
        \end{itemize}
      \item Tada aibės $A$ charakteringoji funkcija būtų:
        \begin{align*}
          \chi_{A}(x) &= \sgi(|f(h(x)) - x|) \\
          &=%
          \begin{cases}
            \sgi(|f(z) - x|), & \text{ kur } f(z) = x, %
              \text{ jei } x \in A\\
            \sgi(|f(z) - x|), & \text{ kur } f(z) \neq x, %
              \text{ jei } x \not \in A
          \end{cases},\\
          &=%
          \begin{cases}
            \sgi(0), & \text{ jei } x \in A \\
            \sgi(y+1), & \text{ jei } x \not \in A
          \end{cases},\\
          &=%
          \begin{cases}
            1, & \text{ jei } x \in A \\
            0, & \text{ jei } x \not \in A
          \end{cases}.
        \end{align*}
      \item $\chi_{A}(x) \in \BR$, nes gauta iš $\BR$ ir $\PR$ funkcijų
        pritaikius kompozicijos operatorių.
      \item Kadangi $\chi_{A}(x) \in \BR$, tai $A$ – rekursyvi 
        $\implies$ prieštara $\implies$ $\overline{A}$ – nėra rekursyviai
        skaiti aibė.
    \end{enumerate}
  \end{proof}
\end{prop}

\section{Ackermann funkcijos.}

Ackerman funkcijos yra:
\begin{align*}
  B_{0}(a, x) &= a + x \\
  B_{1}(a, x) &= a \cdot x \\
  B_{2}(a, x) &= a^{x} \\
  B_{n}(a, x+1) &= B_{n-1}(a, B_{n}(a, x)) \\
  B_{n}(a, 0) &= 1, \text{ kai } n \geq 2
\end{align*}
Jos yra visur apibrėžtos ir algoritmiškai apskaičiuojamos funkcijos, todėl
yra bendrai rekursyvios funkcijos.

Ackermann funkcijos variantas, kai $a=2$, yra funkcija 
$A(n,x) = B_{n}(2, x)$. Jam teisingos lygybės ir nelygybės:
\begin{enumerate}
  \item $A(n+1,x+1) = B_{n+1}(2, x+1) = B_{n}(2, B_{n+1}(2, x)) = %
    A(n, A(n+1, x))$
  \item $A(0,x) = x+2$
  \item $A(1,0) = 0$
  \item $A(n,0) = 1$, kai $n \geq 2$
  \item $A(n,x) \geq 2^{x}$, kai $n \geq 2$ ir $x \geq 1$
  \item $A(n+1,x) > A(n,x)$, kai $n,x \geq 2$
  \item $A(n,x+1) > A(n,x)$, kai $n,x \geq 2$
  \item $A(n+1,x) \geq A(n,x+1)$, kai $n,x \geq 2$
\end{enumerate}

\begin{defn}[Mažoruojanti funkcija]
  Sakysime, kad funkcija $f(x)$ yra mažoruojama\footnote{Žr. DLKŽ.} 
  funkcijos $h(x)$, jei
  yra toks $n_{0} \in \mathbb{N}$, kad $f(x) < h(x)$ su visais $x > n_{0}$.
\end{defn}

\begin{prop}
  \label{fo}
  Jei $f(x)$ yra vieno argumento $\PR$ funkcija, tai egzistuoja toks
  $n\,(n \in \mathbb{N})$, kad
  \[
  f(x) < A(n,x), \text{ kai } x > 2.
  \]
\end{prop}

\begin{prop}
  Egzistuoja tokios funkcijos, kurios yra bendrai rekursyvios, bet nėra
  primityviai rekursyvios.

  \begin{proof}
    Parodysime, kad funkcija $h(x) = A(x,x)$ yra $\BR$, bet nėra $\PR$.

    $h(x) \in \BR$, nes ji yra visur apibrėžta, algoritmiškai 
    apskaičiuojama funkcija.

    Įrodysim, kad $h(x)$ mažoruoja bet kokią $\PR$ vieno argumento 
    funkciją.

    Tegu $f(x) \in \PR$. Iš \ref{fo} teiginio: 
    $\exists n\,(n \in \mathbb{N})$, kad $f(x) < A(n,x)$, kai $x > 2$.

    Tuomet $f(n+x) < A(n,n+x) < A(n+x,n+x) = h(n+x) \implies %
    \exists n_{0} \, (n_{0} \in \mathbb{N})$, kad $f(x) < h(x)$, kai
    $x > n_{0}$ $\implies$ $f(x)$ yra mažoruojama funkcijos $h(x)$.

    Kadangi bet kokia vieno argumento primityviai rekursyvi funkcija yra
    mažoruojama funkcijos $h(x)$, tai funkcija $h(x) \not \in \PR$.
  \end{proof}
  
\end{prop}

\section{Universaliosios funkcijos}

\begin{defn}[Universalioji funkcija]
  Tarkime, kad $A$ yra kuri nors $n$ argumentų funkcijų aibė. 
  Funkciją $F(x_0,x_1,\dotsc,x_n)$ vadinsime aibės A universaliąja, jei
  $A = \{ F(0,x_1,x_2,\dotsc,x_n),F(1,x_1,x_2,\dotsc,x_n),%
  F(2,x_1,x_2,\dotsc,x_n),\dotsc$, tai yra 
  $F(i, x_1,x_2,\dotsc,x_n) \in A (i = 0,1,2,\dotsc)$ ir nesvarbu, kokia
  būtų $f(x_1,x_2,\dotsc,x_n) \in A$, atsiras bent vienas toks natūralusis
  skaičius $i$, kad $f(x_1,x_2,\dotsc,x_n) = F(i,x_1,x_2,\dotsc,x_n)$.
\end{defn}

\begin{prop}
  \begin{enumerate}
    \item Visų $n$ argumentų $\PR$ funkcijų universalioji
      $F(i,x_1,x_2,\dotsc,x_n) \not \in \PR$.
    \item Visų $n$ argumentų $\BR$ funkcijų universalioji
      $F(i,x_1,x_2,\dotsc,x_n) \not \in \BR$.
  \end{enumerate}
  \begin{proof}
    \begin{enumerate}
      \item Tegu $F(i,x_1,x_2,\dotsc,x_n) \in \PR$.
        Imkime funkciją 
        $g(x_1,x_2,\dotsc,x_n) = F(x_1,x_1,x_2,\dotsc,x_n)+1$.
        $g(x_1,x_2,\dotsc,x_n) \in \PR$, nes gauta iš $\PR$ funkcijų
        $F(i,x_1,x_2,\dotsc,x_n)$ ir $(x+y)$ pritaikius kompozicijos
        operatorių. Kadangi $g(x_1,x_2,\dotsc,x_n)$ yra $n$ argumentų
        $\PR$ funkcija, tai $\exists j\, (j \in \mathbb{N})$ toks, kad
        $F(j,x_1,x_2,\dotsc,x_n) = g(x_1,x_2,\dotsc,x_n)$. Tuomet
        paimkime $x_1 = x_2 = \cdots = x_n = j$, tada 
        $F(j, j, j, \dotsc, j) = g(j, j, \dotsc, j) =%
        F(j, j, j, \dotsc, j) + 1 \implies F(j, j, j, \dotsc, j)$ – 
        neapibrėžta $\implies$ prieštara, nes visos $\PR$ funkcijos
        yra apibrėžtos su visais argumentais, todėl mūsų prielaida, jog
        $F(i,x_0,x_1,\dotsc,x_n) \in \PR$ yra neteisinga.

      \item Tegu $F(i,x_1,x_2,\dotsc,x_n) \in \BR$.
        Imkime funkciją 
        $g(x_1,x_2,\dotsc,x_n) = F(x_1,x_1,x_2,\dotsc,x_n)+1$.
        $g(x_1,x_2,\dotsc,x_n) \in \BR$, nes gauta iš $\BR$ funkcijų
        $F(i,x_1,x_2,\dotsc,x_n)$ ir $(x+y)$ pritaikius kompozicijos
        operatorių. Kadangi $g(x_1,x_2,\dotsc,x_n)$ yra $n$ argumentų
        $\BR$ funkcija, tai $\exists j\, (j \in \mathbb{N})$ toks, kad
        $F(j,x_1,x_2,\dotsc,x_n) = g(x_1,x_2,\dotsc,x_n)$. Tuomet
        paimkime $x_1 = x_2 = \cdots = x_n = j$, tada 
        $F(j, j, j, \dotsc, j) = g(j, j, \dotsc, j) =%
        F(j, j, j, \dotsc, j) + 1 \implies F(j, j, j, \dotsc, j)$ – 
        neapibrėžta $\implies$ prieštara, nes visos $\BR$ funkcijos
        yra apibrėžtos su visais argumentais, todėl mūsų prielaida, jog
        $F(i,x_0,x_1,\dotsc,x_n) \in \BR$ yra neteisinga.
    \end{enumerate}
  \end{proof}
\end{prop}

\begin{prop}
  Visų vieno argumento $\PR$ funkcijų aibei egzistuoja universalioji
  funkcija, kuri yra $\PR$.
  \begin{proof}
    Pagal \ref{pr1arg} teiginį visas vieno argumento $\PR$ funkcijas
    galime išreikšti per bazines $s(x)$, $q(x)$ ir sudėties, kompozicijos,
    bei iteracijos operatorius.
    Remiantis išraiška per $s(x)$ ir $q(x)$, visoms $\PR$ vieno argumento
    funkcijoms suteiksime numerius tokiu būdu:
    \begin{itemize}
      \item $n(s(x)) = 1$
      \item $n(q(x)) = 3$
    \end{itemize}
    Jei $n(f(x)) = a$ ir $n(g(x)) = b$, tai
    \begin{itemize}
      \item $h(x) = f(x) + g(x) \implies n(h(x)) = 2 \cdot 3^a \cdot 5^b$
      \item $h(x) = f(g(x)) \implies n(h(x)) = 4 \cdot 3^a \cdot 5^b$
      \item $h(x) = f(x)^{i} \implies n(h(x)) = 8 \cdot 3^a$
    \end{itemize}
    Tarkime, turime funkciją $f_{n}(x)$, kurios numeris yra $n$. 
    Pažymėkime $F(n,x) = f_{n}(x)$. Tada:
    \[
    F(n,x) =%
    \begin{cases}
      f_{a}(x) + f_{b}(x), & \text{ jei } n = 2\cdot3^a\cdot5^b \\
      f_{a}(f_{b}(x)), & \text{ jei } n = 4\cdot3^a\cdot5^b \\
      f_{a}(f_{n}(x-1)), & \text{ jei } n = 8\cdot3^a \text{ ir } x > 0 \\
      0, & \text{ jei } n = 8\cdot3^a \text{ ir } x = 0 \\
      s(x), & \text{ jei } n = 1 \\
      q(x), & \text{ jei } n = 3
    \end{cases}
    \]
    $F(n,x)$ yra algoritmiškai apskaičiuojama. Tada visų vieno argumento
    $\PR$ universalioji yra:
    \[
    D(n,x) =%
    \begin{cases}
      F(n,x), & \text{ jei $n$ – kažkurios funkcijos numeris} \\
      0, & \text{ kitu atveju }
    \end{cases}.
    \]
    $D(n,x) \in \BR$, nes yra algoritmiškai apskaičiuojama ir visur 
    apibrėžta.
  \end{proof}
\end{prop}

\begin{prop}
  Visų $n$ argumentų $\PR$ funkcijų universalioji yra funkcija:
  \[
  D^{n+1}(x_0,x_1,x_2,\dotsc,x_n) = D(x_0,\alpha_n(x_1,x_2,\dotsc,x_n))
  \]
\end{prop}

\begin{defn}
  $\DR$ $n$ argumentų funkcijos grafikas yra taškų, sudarytų iš 
  $(n+1)$ elemento, aibė:
  \[
  G = \{ (x_1,x_2,\dotsc,x_n,y): f(x_1,x_2,\dotsc,x_n) = y \}
  \]
\end{defn}


                          % Rekursyviosios funkcijos.
% \input{08.tex}                          % Lambda skaičiavimas.
% \input{09.tex}                          % Rekursyviai skaičios ir 
%                                         % rekursyvios aibės.
% \input{10.tex}                          % Universaliosios funkcijos.

\end{document}
