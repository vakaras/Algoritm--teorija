\chapter{Sekvencinis skaičiavimas}

\begin{defn}[Sekvencija]
  Tokia išraiška $\underbrace{F_{1},F_{2},\dots,F_{n}}_{anticedentas}%
  \vdash \underbrace{G_{1},G_{2},\dots,G_{m}}_{sukcedentas}$, jei 
  \mbox{$n + m > 0$}.
  \begin{note}
    \hfill \\
    \begin{itemize}
      \item $\vdash F$ – formulė yra tapačiai teisinga.
      \item $A_{1},A_{2},\dots,A_{n} \vdash F$ – iš prielaidų seka išvada F.
      \item $A_{1},A_{2},\dots,A_{n} \vdash B_{1},B_{2},\dots,B_{m}$ – iš 
        prielaidų seka bent viena iš išvadų $B_{1},\dots,B_{m}$.
      \item $F \vdash$ – formulė yra tapačiai klaidinga.
    \end{itemize} 
  \end{note}
\end{defn}

\begin{defn}[Sekvencinis skaičiavimas G]
  Skaičiavimas su aksioma $\Gamma',A,\Gamma'' \vdash \Delta',A,\Delta''$
  ir taisyklėmis ($\Gamma,\Gamma',\Gamma'',\Delta,\Delta',\Delta''$ – 
  formulių sekos, $A,B$ – formulės):
  \begin{description}
    \item[$(\vdash \neg)$] 
      \[
      \begin{array}{c}
        \Gamma',A,\Gamma'' \vdash \Delta',\Delta'' \\
        \cline{1-1}
        \Gamma',\Gamma'' \vdash \Delta',\neg A,\Delta''
      \end{array}
      \]
    \item[$(\neg \vdash)$] 
      \[
      \begin{array}{c}
        \Gamma',\Gamma'' \vdash \Delta',A,\Delta'' \\
        \cline{1-1}
        \Gamma',\neg A,\Gamma'' \vdash \Delta',\Delta''
      \end{array}
      \]
    \item[$(\vdash \lor)$] 
      \[
      \begin{array}{c}
        \Gamma',\Gamma'' \vdash \Delta',A,B,\Delta'' \\
        \cline{1-1}
        \Gamma',\Gamma'' \vdash \Delta',A \lor B,\Delta''
      \end{array}
      \]
    \item[$(\lor \vdash)$] 
      \[
      \begin{array}{c}
        \Gamma',A,\Gamma'' \vdash \Delta',\Delta'' \qquad %
        \Gamma',B,\Gamma'' \vdash \Delta',\Delta'' \\
        \cline{1-1}
        \Gamma',A \lor B,\Gamma'' \vdash \Delta',\Delta''
      \end{array}
      \]
    \item[$(\vdash \land)$] 
      \[
      \begin{array}{c}
        \Gamma',\Gamma'' \vdash \Delta',A,\Delta'' \qquad %
        \Gamma',\Gamma'' \vdash \Delta',B,\Delta'' \\
        \cline{1-1}
        \Gamma',\Gamma'' \vdash \Delta',A \land B, \Delta''
      \end{array}
      \]
    \item[$(\land \vdash)$] 
      \[
      \begin{array}{c}
        \Gamma',A,B,\Gamma'' \vdash \Delta',\Delta'' \\
        \cline{1-1}
        \Gamma',A \land B,\Gamma'' \vdash \Delta',\Delta''
      \end{array}
      \]
    \item[$(\vdash \to)$] 
      \[
      \begin{array}{c}
        \Gamma',A,\Gamma'' \vdash \Delta',B,\Delta'' \\
        \cline{1-1}
        \Gamma',\Gamma'' \vdash \Delta',A \to B,\Delta''
      \end{array}
      \]
    \item[$(\to \vdash)$] 
      \[
      \begin{array}{c}
        \Gamma',\Gamma'' \vdash \Delta',A,\Delta'' \qquad %
        \Gamma',B,\Gamma'' \vdash \Delta',\Delta'' \\
        \cline{1-1}
        \Gamma',A \to B,\Gamma'' \vdash \Delta',\Delta''
      \end{array}
      \]
  \end{description}
\end{defn}

Sakysime, kad sekvencija $F_{1},\dots,F_{n} \vdash G_{1},\dots,G_{m}$ yra
išvedame skaičiavime G, jei galime sukonstruoti tokį medį (grafą),
kurio visose viršūnėse yra sekvencijos, šaknyje yra pradinė sekvencija
$F_{1},\dots,F_{n} \vdash G_{1},\dots,G_{m}$ ir kiekvienai viršūnei 
teisinga: jei viršūnės $N$, kurioje 
yra sekvencija S, vaike(-uose) $N_{1}$ (ir $N_{2}$) yra sekvencija(-os)
$S_{1}$ (ir $S_{2}$), tai sekvencija S yra gauta iš sekvencijos(-jų)
$S_{1}$ (ir $S_{2}$) pagal kažkurią taisyklę, ir kurio (medžio)
visuose lapuose yra aksiomos.

Sakysime, kad taisyklė yra \emph{apverčiama}, jei teisinga: jei taisyklės
išvada išvedama, tai išvedamos ir visos taisyklės prielaidos.

Visos sekvencinio skaičiavimo G taisyklės yra apverčiamos.

Jei išvedant gautas lapas, kuriame nėra nei vienos formulės ir jis nėra
aksioma, tai reiškia kad:
\begin{enumerate}
  \item sekvencija yra neišvedama, arba
  \item skaičiavime buvo panaudota neapverčiama taisyklė.
\end{enumerate}

\begin{prop}
  Formulė F yra tapačiai teisinga tada ir tik tada, kai sekvenciniame
  skaičiavime G išvedama sekvencija $\vdash F$.
\end{prop}
