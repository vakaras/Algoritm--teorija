\chapter{Turingo mašinos ir jų variantai}

\begin{defn}[Determinuota vienajuostė Turingo mašina]
  Ketvertas $<\Sigma,Q,F,\delta>$, kur:
  \begin{description}
    \item[$\Sigma$] – baigtinė aibė – abėcėlė. (Jei nepaminėta kitaip:
      $\Sigma = \{0, 1, b\}$.)
    \item[$Q$] – baigtinė aibė – būsenų aibė. 
      ($Q = \{q_0,q_1,\dotsc,q_n\}$, kur $q_0$ – pradinė būsena.)
    \item[$F$] – galutinių būsenų aibė ($F \subset Q$).
    \item[$\delta$] – perėjimų funkcija. 
      ($\delta: Q \times \Sigma \to Q \times \Sigma \times \{K,N,D\}$)
  \end{description}
\end{defn}

\begin{defn}[Apibrėžta \emph{TM}]
  Sakysime, jog \emph{TM} su pradiniais duomenimis $X$ yra apibrėžta, jei 
  pradžioje į duomenų juostą įrašius žodį $X$, \emph{TM} po baigtinio
  žingsnių kiekio patenka į vieną iš galutinių būsenų.
\end{defn}

\begin{defn}[Turingo mašina apskaičiuoja funkciją]
  Sakysime, kad \emph{TM} $M$ apskaičiuoja funkciją 
  $f(x_1,x_2,\dotsc,x_n)$, jei egzistuoja toks kodavimas abėcėlės 
  $\Sigma$ simboliais $\cod(x_1,x_2,\dotsc,x_n) = \tilde{x}$, kuriam
  teisinga:
  \begin{enumerate}
    \item jei $f(x_1,x_2,\dotsc,x_n) = y$ (funkcija apibrėžta), tada 
      \emph{TM} $M$ su pradiniais duomenimis $\cod(x_1,x_2,\dotsc,x_n)$
      yra apibrėžta ir baigus darbą, į dešinę nuo skaitymo galvutės yra
      žodis $\cod(y)$;
    \item jei $f(x_1,x_2,\dotsc,x_n)$ – neapibrėžta, tada \emph{TM} $M$
      su pradiniais duomenimis $\cod(x_1,x_2,\dotsc,x_n)$ irgi yra
      neapibrėžta.
  \end{enumerate}
\end{defn}

\begin{defn}[Determinuota m-juostė \emph{TM}]
  Ketvertas $<\Sigma,Q,F,\delta>$, kur:
  \begin{description}
    \item[$\Sigma$] – baigtinė aibė – abėcėlė.
    \item[$Q$] – baigtinė būsenų aibė.
    \item[$F$] – galutinių būsenų aibė. ($F \subset Q$)
    \item[$\delta$] – perėjimų funkcija. \\
      ($\delta: Q \times \underbrace{\Sigma \times \Sigma \times \cdots%
        \times \Sigma}_{m} \to Q \times \underbrace{%
          \Sigma \times \Sigma \times \cdots \times \Sigma}_{m}%
        \times \underbrace{
          \{K,D,N\} \times \{K,D,N\} \times \cdots \times \{K,D,N\}}_{m}$)
  \end{description}
\end{defn}

\begin{defn}[Nedeterminuota Turingo mašina]
  Turingo mašina, kurios perėjimų funkcija yra daugiareikšmė.
\end{defn}


\section{Baigtiniai automatai}

\begin{defn}[Baigtinis automatas]
  Vienajuostė determinuota Turiningo mašina, kurios perėjimu funkcija
  yra $\delta(q_i, a) = (q_j, a, D)$ ir kuri (\emph{TM}), baigia darbą
  tada, kai pasiekia pirmąją tuščia ląstelę.
\end{defn}

\begin{defn}[Baigtinio automato kalba]
  Tokia žodžių, su kuriais automatas baigia darbą vienoje iš galutinių 
  būsenų, aibė.
\end{defn}

\begin{prop}
  Baigtinė aibė yra baigtinio automato kalba.
\end{prop}

\begin{prop}
  Baigtinio automato kalbos $A$ papildinys $\bar{A}$ irgi yra baigtinio
  automato kalba.
\end{prop}

\begin{prop}
  Jei $A_1$ ir $A_2$ yra baigtinio automato kalbos, tai ir 
  $A_1 \cup A_2$, bei $A_1 \cap A_2$ yra baigtinio automato kalbos.

  \begin{proof}
    Grafų $G_1$ ir $G_2$ Dekarto sandauga yra grafas, kurio viršūnės yra 
    visos įmanomos poros $(q_i,q_j)$, kur $q_i$ yra $G_1$ viršūnė, o 
    $q_j$ – $G_2$ viršūnė. Iš viršūnės $(q_i,q_j)$ eis briauna į 
    viršūnę $(q_k,q_l)$ su simboliu $a$, jei grafe $G_1$ eina briauna iš
    viršūnės $q_i$ į viršūnę $q_k$ su simboliu $a$ ir grafe $G_2$ eina
    briauna iš viršūnės $q_j$ į viršūnę $q_l$ su simboliu $a$.
    
    $F_{A \cup B}$ – visos viršūnės $(q_i,q_j)$, kur $q_i \in F_{A}$
    arba $q_j \in F_{B}$.

    $F_{A \cap B}$ – visos viršūnės $(q_i,q_j)$, kur $q_i \in F_{A}$
    ir $q_j \in F_{B}$.
  \end{proof}
\end{prop}

\begin{prop}
  Baigtinių automatų kalbų
  \begin{enumerate}
    \item konkatenacija $:= \{ uv : u \in A_1, v \in A_2 \}$, kur 
      $A_1$, $A_2$ – automato kalbos;
    \item iteracija $:=\{u_1 u_2 \dots u_k : u_i \in A_i, i=1,2,\dotsc,k\}$,
      kur $A_1$ – automato kalba;
    \item atspindys $:=\{a_1 a_2 \dots a_n : a_n a_{n-1} \dots a_1 %
      \in A_1 \}$, kur $A_1$ – automato kalba
  \end{enumerate}
    irgi yra baigtinių automatų kalbos.
\end{prop}

\section{Algoritmų sudėtingumas}

\begin{notation}
  \hfill \\
  \begin{description}
    \item[$i(v)$] – žodžio $v$ ilgis;
    \item[$t(v)$] – žingsnių kiekis, kurį atlieka \emph{TM}, jei 
      pradinių duomenų juostoje yra žodis $v$.
    \item[$s(v)$] – panaudotų ląstelių kiekis, kurį sunaudojo \emph{TM},
      jei pradinių duomenų juostoje buvo žodis $v$.
  \end{description}
\end{notation}

\begin{defn}[Turingo mašinos M sudėtingumas laiko atžvilgiu]
  Funkcija: $T_{M} (n) = \max \{ t(v) : i(v) = n \}$.
\end{defn}

\begin{defn}[Turingo mašinos M sudėtingumas atminties atžvilgiu]
  Funkcija: $S_{M} (n) = \max \{ s(v) : i(v) = n \}$.
\end{defn}

\begin{defn}[Turingo mašinos kalba]
  Žodžių, su kuriais \emph{TM} baigia darbą galutinėje būsenoje, aibė.
\end{defn}

\begin{defn}[Problema (aibė) išsprendžiama su \emph{TM}]
  Sakysime, jog problema (aibė) A yra išsprendžiama su Turingo mašina,
  jei yra tokia Turingo mašina, kurios kalba yra A.
\end{defn}

Sakysime, kad \emph{TM} sudėtingumas atminties atžvilgiu yra 
$f(n)$, jei $\exists c : S_{M}(n) = cf(n), \forall n$.

Sakysime, kad \emph{TM} sudėtingumas laiko atžvilgiu yra 
$f(n)$, jei $\exists c : T_{M}(n) = cf(n), \forall n$.

\begin{defn}[Sudėtingumo klasės]
  \hfill \\
  \begin{itemize}
    \item $DTIME(f(n))$ – sudėtingumo klasė, kuriai priklauso visi
      uždaviniai, kuriems egzistuoja juos sprendžianti daugiajuostė 
      determinuota \emph{TM}, kurios sudėtingumas laiko atžvilgiu 
      yra $f(n)$.
    \item $NTIME(f(n))$ – sudėtingumo klasė, kuriai priklauso visi
      uždaviniai, kuriems egzisutoja juos sprendžianti daugiajuostė 
      nedeterminuota \emph{TM}, kurios sudėtingumas laiko atžvilgiu
      yra $f(n)$.
    \item $DSPACE(f(n))$ – sudėtingumo klasė, kuriai priklauso visi
      uždaviniai, kuriems egzistuoja juos sprendžianti daugiajuostė
      determinuota \emph{TM}, kurios sudėtingumas atminties atžvilgiu
      yra $f(n)$.
    \item $NSPACE(f(n))$ – sudėtingumo klasė, kuriai priklauso visi
      uždaviniai, kuriems egzsituoja juos sprendžianti daugiajuostė
      nedeterminuota \emph{TM}, kurios sudėtingumas atminties atžvilgiu
      yra $f(n)$.
  \end{itemize}
\end{defn}

\begin{defn}[Sudėtingumo klasės]
  \hfill \\
  \begin{itemize}
    \item $L = DSPACE(\log n)$
    \item $NL = NSPACE(\log n)$
    \item $P = DTIME(n^k)$, kur $k$ – konstanta.
    \item $NP = NTIME(n^k)$, kur $k$ – konstanta.
    \item $PSPACE = DSPACE(n^k)$, kur $k$ – konstanta.
    \item $EXP = DTIME(2^{n^{k}})$, kur $k$ – konstanta.
  \end{itemize}
\end{defn}

\begin{prop}
  $L \subseteq NL \subseteq P \subseteq NP %
    \subseteq PSPACE \subseteq EXP$
  ir $NL \not = PSPACE$ ir $P \not = EXP$.
\end{prop}
